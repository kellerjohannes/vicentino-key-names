%% Auto-generated file: 2023-08-08T22:42:56.000000+10:00
\documentclass[10pt,landscape,DIV=17,a4paper]{scrartcl}
\usepackage[utf8]{inputenc}
\usepackage[T1]{fontenc}
\usepackage[ngerman]{babel}
\usepackage{graphicx}
\usepackage{longtable}
\usepackage{wrapfig}
\usepackage{tipa}
\usepackage{array}
\usepackage{relsize}
\usepackage{amssymb}
\usepackage{mathtools}
\usepackage{wasysym}
\usepackage{booktabs}
\usepackage{soul}
\usepackage{titling}
\usepackage{tikz}
\setcounter{secnumdepth}{0}

\usepackage{newunicodechar}
\newunicodechar{♮}{$\natural$}
\newunicodechar{♭}{$\flat$}
\newunicodechar{♯}{$\sharp$}
\newunicodechar{➚}{{\small$\nearrow$}}
\newunicodechar{➘}{{\small$\searrow$}}
\newunicodechar{Ȧ}{\.A}
\newunicodechar{Ḃ}{\.B}
\newunicodechar{Ċ}{\.C}
\newunicodechar{Ḋ}{\.D}
\newunicodechar{Ė}{\.E}
\newunicodechar{Ḟ}{\.F}
\newunicodechar{Ġ}{\.G}
\newunicodechar{ʼ}{'}

\def\nsharp#1{#1$\sharp$}
\def\nflat#1{#1$\flat$}
\def\nnatural#1{#1$\natural$}
\def\ndot#1{\.{#1}}
\def\nnaturaldot#1{\.{#1}$\natural$}
\def\ncomma#1{\'{#1}}
\def\nnaturalcomma#1{\'{#1}$\natural$}
\def\nflatdot#1{\.{#1}$\flat$}
\def\nsharpdot#1{\.{#1}$\sharp$}


%% This is used for a thighter box around key names
\setlength\fboxsep{1.2pt}

\def\typesetInterval#1#2#3{\small{$\lvert$#1#2#3$\rvert$}}
\def\typesetKey#1#2{\fbox{\footnotesize{\textsc{#1#2}}}}
\def\typesetLinecounter#1{\tiny{\textsc{#1}}}
\def\typesetTag#1{\texttt{#1}}

\renewcommand*{\maketitle}{\noindent%
\parbox{\dimexpr\linewidth-2\fboxsep}{\centering%
\fontsize{20}{24}\selectfont\sffamily\bfseries\thetitle\\[1ex]%
\fontsize{12}{14}\selectfont\centering\today\hspace{1cm}\theauthor}}


\newcolumntype{C}[1]{>{\centering\arraybackslash}p{#1}}

\renewcommand{\arraystretch}{1.3}
\author{Johannes Keller}
\date{\today}
\subtitle{Berücksichtigt sämtliche Tastennamen, Intervalle und Noten der Kapitel b5-c8 bis b5-c38.}

\title{Inventar aller Probanden, die \typesetTag{shorthand}-Notation kennen\\\relsize{-3}Berücksichtigt Tastennamen, Intervalle und Noten der Kapitel b5-c8 bis b5-c38}

\begin{document}

\maketitle

\begin{center}

\vspace{3ex}

{\large{Sämtliche Tastennamen mit \typesetTag{:note-name} B♯, E♯, Ċ, Ḟ und Cʼ, sortiert nach \typesetTag{:note-name} und Vorhandensein von \typesetTag{:regular-shorthand}.}}

\vspace{2ex}

{\footnotesize{Legende:
\#~Zeilennummerierung,
T~Objekttyp ($\Square$~Taste, $\CIRCLE$~Note, {\tiny$\Square$$\Square$}~Intervall zwischen Tasten, {\tiny$\CIRCLE$$\CIRCLE$}~Intervall zwischen Noten),
I~Objekt-ID,
B~\emph{libro},
C~\emph{capitolo},
»\texttt{D}«~\texttt{:diplomatic},
»\texttt{sh}«~\texttt{:regular-shorthand},
»\texttt{C}«~\texttt{:obvious-correction},
»\texttt{R}«~\texttt{:recommended-correction},
»\texttt{om}«~\texttt{:omitted-text},
»\texttt{extd}«~\texttt{:extended-key},
»\texttt{qs}«~\texttt{:quintenschaukel},
»\texttt{p}«~\texttt{:propinqua},
»\texttt{ip}«~\texttt{:inverse-propinqua},
»\texttt{$\neg$ip}«~\texttt{:avoid-inverse-propinqua},
»\texttt{pp}«~\texttt{:propinquissima},
»\texttt{ipp}«~\texttt{:inverse-propinquissima},
»\texttt{$\neg$ipp}«~\texttt{:avoid-inverse-propinquissima},
»\texttt{ex}«~\texttt{:exotic},
»\texttt{$\neg$ex}«~\texttt{:avoid-exotic},
»\texttt{{\small\fbox{7}}}«~\texttt{:septimal},
Skala der Intervallgrössen: Markierungen für 1:1 81:80, 128:125, 6:5, 5:4, 3:2, 8:5, 5:3 und 2:1.
}}

\begin{longtable}{p{1.5mm}C{1.5mm}p{4.5mm}p{1mm}p{2mm}p{6.5cm}p{15mm}p{1cm}p{11cm}}

\toprule
\# &
\emph{T} &
\emph{I} &
\emph{B} &
\emph{C} &
\emph{Name (normalisierte Orthographie)} &
&
\emph{Tags} &
\emph{Kommentar}\\
\midrule
\endhead


\typesetLinecounter{1} & $\Square$ & 38 & 5 & 8 & Cfaut secondo in terzo ordine & B♯ \typesetKey{C}{3} & {\footnotesize\texttt{D} } &  \\
\typesetLinecounter{2} & $\Square$ & 41 & 5 & 8 & secondo Cfaut nel terzo ordine & B♯ \typesetKey{C}{3} & {\footnotesize\texttt{D} } &  \\
\typesetLinecounter{3} & $\Square$ & 43 & 5 & 8 & Cfaut secondo nel terzo ordine & B♯ \typesetKey{C}{3} & {\footnotesize\texttt{D} } &  \\
\typesetLinecounter{4} & $\Square$ & 44 & 5 & 8 & Cfaut terzo & B♯ \typesetKey{C}{3} & {\footnotesize\texttt{D} } & Hier als B♯ verstanden, demonstriert die Versuchung, diese Taste
\emph{terzo} zu nennen, weil sie im \emph{terzo ordine} liegt, was aber falsch
ist, denn sie ist die erste Taste nach dem Cfaut, also das \emph{Cfaut
secondo} (\emph{nel terzo ordine}).
 \\
\typesetLinecounter{5} & $\Square$ & 141 & 5 & 9 & Cfaut secondo nel terzo ordine & B♯ \typesetKey{C}{3} & {\footnotesize\texttt{D} } &  \\
\typesetLinecounter{6} & $\Square$ & 157 & 5 & 9 & Csolfaut secondo in terzo ordine & B♯ \typesetKey{C}{3} & {\footnotesize\texttt{D} } &  \\
\typesetLinecounter{7} & $\Square$ & 235 & 5 & 10 & Cfaut secondo in terzo ordine & B♯ \typesetKey{C}{3} & {\footnotesize\texttt{D} } &  \\
\typesetLinecounter{8} & $\Square$ & 246 & 5 & 10 & Csolfaut secondo in terzo ordine & B♯ \typesetKey{C}{3} & {\footnotesize\texttt{D} } &  \\
\typesetLinecounter{9} & $\Square$ & 512 & 5 & 12 & Cfaut secondo in terzo ordine & B♯ \typesetKey{C}{3} & {\footnotesize\texttt{D} } &  \\
\typesetLinecounter{10} & $\Square$ & 520 & 5 & 12 & Cfaut secondo nel terzo ordine & B♯ \typesetKey{C}{3} & {\footnotesize\texttt{D} } &  \\
\typesetLinecounter{11} & $\Square$ & 956 & 5 & 17 & Csolfaut secondo in terzo ordine & B♯ \typesetKey{C}{3} & {\footnotesize\texttt{C} } & Original: »Csolfaut terzo«. Es handelt sich hier um eine unregelmässige Abkürzung der Bezeichnung von B♯. \\
\typesetLinecounter{12} & $\Square$ & 1399 & 5 & 22 & Csolfaut secondo in terzo ordine & B♯ \typesetKey{C}{3} & {\footnotesize\texttt{D} } &  \\
\typesetLinecounter{13} & $\Square$ & 1501 & 5 & 23 & Csolfaut secondo in terzo ordine & B♯ \typesetKey{C}{3} & {\footnotesize\texttt{D} } &  \\
\typesetLinecounter{14} & $\Square$ & 1536 & 5 & 23 & Csolfaut secondo in terzo ordine & B♯ \typesetKey{C}{3} & {\footnotesize\texttt{D} } &  \\
\typesetLinecounter{15} & $\Square$ & 1695 & 5 & 25 & Csolfaut secondo in terzo ordine & B♯ \typesetKey{C}{3} & {\footnotesize\texttt{D} } &  \\
\typesetLinecounter{16} & $\Square$ & 2115 & 5 & 29 & Csolfaut acuto secondo in terzo ordine & B♯ \typesetKey{C}{3} & {\footnotesize\texttt{D} } &  \\
\typesetLinecounter{17} & $\Square$ & 2357 & 5 & 32 & Csolfaut secondo in terzo ordine & B♯ \typesetKey{C}{3} & {\footnotesize\texttt{D} } &  \\
\typesetLinecounter{18} & $\Square$ & 2358 & 5 & 32 & Csolfaut acuto secondo in terzo ordine & B♯ \typesetKey{C}{3} & {\footnotesize\texttt{D} } &  \\
\typesetLinecounter{19} & $\Square$ & 2399 & 5 & 32 & Csolfaut [secondo in terzo ordine] acuto & B♯ \typesetKey{C}{3} & {\footnotesize\texttt{C} } & Original: »Csolfaut acuto«. Tastenbezeichnung ergibt sich eindeutig aus dem Kontext. \\
\typesetLinecounter{20} & $\Square$ & 46 & 5 & 8 & Cfaut secondo [Cfaut secondo in terzo ordine] & B♯ \typesetKey{C}{3} & {\footnotesize\texttt{D} \texttt{sh} } & Reguläre Verkürzung von \emph{Cfaut secondo in terzo ordine}. \\
\typesetLinecounter{21} & $\Square$ & 1028 & 5 & 18 & Cfaut secondo [Cfaut secondo nel terzo ordine] & B♯ \typesetKey{C}{3} & {\footnotesize\texttt{D} \texttt{sh} } &  \\
\typesetLinecounter{22} & $\Square$ & 1224 & 5 & 20 & Csolfaut secondo [Csolfaut secondo nel terzo ordine] & B♯ \typesetKey{C}{3} & {\footnotesize\texttt{D} \texttt{sh} } &  \\
\typesetLinecounter{23} & $\Square$ & 1307 & 5 & 21 & Csolfaut secondo [Csolfaut secondo in terzo ordine] & B♯ \typesetKey{C}{3} & {\footnotesize\texttt{D} \texttt{sh} } &  \\
\typesetLinecounter{24} & $\Square$ & 1437 & 5 & 22 & Csolfaut secondo [Csolfaut secondo in terzo ordine] & B♯ \typesetKey{C}{3} & {\footnotesize\texttt{D} \texttt{sh} } &  \\
\typesetLinecounter{25} & $\Square$ & 1724 & 5 & 25 & Csolfaut secondo [Csolfaut secondo in terzo ordine] & B♯ \typesetKey{C}{3} & {\footnotesize\texttt{D} \texttt{sh} } &  \\
\typesetLinecounter{26} & $\Square$ & 2356 & 5 & 32 & Csolfaut secondo [Csolfaut secondo in terzo ordine] & B♯ \typesetKey{C}{3} & {\footnotesize\texttt{D} \texttt{sh} } &  \\
\typesetLinecounter{27} & $\Square$ & 2378 & 5 & 32 & Cfaut secondo [Csolfaut secondo in terzo ordine] & B♯ \typesetKey{C}{3} & {\footnotesize\texttt{D} \texttt{sh} } &  \\
\typesetLinecounter{28} & $\Square$ & 2379 & 5 & 32 & medesimo Cfaut ascendente [Csolfaut secondo in terzo ordine] & B♯ \typesetKey{C}{3} & {\footnotesize\texttt{D} \texttt{sh} } &  \\
\typesetLinecounter{29} & $\Square$ & 2400 & 5 & 32 & Csolfaut secondo [Csolfaut secondo in terzo ordine] & B♯ \typesetKey{C}{3} & {\footnotesize\texttt{D} \texttt{sh} } &  \\
\typesetLinecounter{30} & $\Square$ & 2401 & 5 & 32 & Cfaut secondo [Csolfaut secondo in terzo ordine] & B♯ \typesetKey{C}{3} & {\footnotesize\texttt{D} \texttt{sh} } &  \\
\typesetLinecounter{31} & $\Square$ & 2445 & 5 & 33 & Csolfaut secondo acuto [Csolfaut secondo in terzo ordine] & B♯ \typesetKey{C}{3} & {\footnotesize\texttt{D} \texttt{sh} } &  \\
\typesetLinecounter{32} & $\Square$ & 1344 & 5 & 21 & Csolfaut quarto in sesto ordine & Cʼ \typesetKey{C}{6} & {\footnotesize\texttt{D} \texttt{extd} } &  \\
\typesetLinecounter{33} & $\Square$ & 1443 & 5 & 22 & Csolfaut quarto in sesto ordine & Cʼ \typesetKey{C}{6} & {\footnotesize\texttt{D} \texttt{extd} } &  \\
\typesetLinecounter{34} & $\Square$ & 1598 & 5 & 24 & Csolfaut sesto [Csolfaut quarto in sesto ordine] & Cʼ \typesetKey{C}{6} & {\footnotesize\texttt{D} \texttt{sh} \texttt{extd} } &  \\
\typesetLinecounter{35} & $\Square$ & 1635 & 5 & 24 & Csolfaut sesto [Csolfaut quarto in sesto ordine] & Cʼ \typesetKey{C}{6} & {\footnotesize\texttt{D} \texttt{extd} \texttt{sh} } &  \\
\typesetLinecounter{36} & $\Square$ & 25 & 5 & 8 & Ffaut secondo in terzo ordine & E♯ \typesetKey{F}{3} & {\footnotesize\texttt{D} } &  \\
\typesetLinecounter{37} & $\Square$ & 129 & 5 & 9 & Ffaut secondo in terzo ordine & E♯ \typesetKey{F}{3} & {\footnotesize\texttt{D} } &  \\
\typesetLinecounter{38} & $\Square$ & 225 & 5 & 10 & Ffaut secondo in terzo ordine & E♯ \typesetKey{F}{3} & {\footnotesize\texttt{D} } &  \\
\typesetLinecounter{39} & $\Square$ & 257 & 5 & 10 & Ffaut secondo in terzo ordine & E♯ \typesetKey{F}{3} & {\footnotesize\texttt{D} } &  \\
\typesetLinecounter{40} & $\Square$ & 502 & 5 & 12 & Ffaut secondo in terzo ordine & E♯ \typesetKey{F}{3} & {\footnotesize\texttt{D} } &  \\
\typesetLinecounter{41} & $\Square$ & 530 & 5 & 12 & Ffaut secondo in terzo ordine & E♯ \typesetKey{F}{3} & {\footnotesize\texttt{D} } &  \\
\typesetLinecounter{42} & $\Square$ & 971 & 5 & 17 & Ffaut secondo in terzo ordine & E♯ \typesetKey{F}{3} & {\footnotesize\texttt{D} } &  \\
\typesetLinecounter{43} & $\Square$ & 1234 & 5 & 20 & Ffaut secondo nel terzo ordine & E♯ \typesetKey{F}{3} & {\footnotesize\texttt{C} } & Original: »Ffaut terzo acuto«. \\
\typesetLinecounter{44} & $\Square$ & 1805 & 5 & 26 & Ffaut secondo in terzo ordine & E♯ \typesetKey{F}{3} & {\footnotesize\texttt{D} } &  \\
\typesetLinecounter{45} & $\Square$ & 1894 & 5 & 27 & Ffaut secondo in terzo ordine & E♯ \typesetKey{F}{3} & {\footnotesize\texttt{D} } &  \\
\typesetLinecounter{46} & $\Square$ & 1909 & 5 & 27 & Ffaut secondo in terzo ordine & E♯ \typesetKey{F}{3} & {\footnotesize\texttt{D} } &  \\
\typesetLinecounter{47} & $\Square$ & 1991 & 5 & 28 & Ffaut secondo in terzo ordine & E♯ \typesetKey{F}{3} & {\footnotesize\texttt{D} } &  \\
\typesetLinecounter{48} & $\Square$ & 2004 & 5 & 28 & Ffaut secondo in terzo ordine & E♯ \typesetKey{F}{3} & {\footnotesize\texttt{D} } &  \\
\typesetLinecounter{49} & $\Square$ & 2184 & 5 & 30 & Ffaut secondo in terzo ordine & E♯ \typesetKey{F}{3} & {\footnotesize\texttt{D} } &  \\
\typesetLinecounter{50} & $\Square$ & 2191 & 5 & 30 & Ffaut secondo in terzo ordine & E♯ \typesetKey{F}{3} & {\footnotesize\texttt{D} } &  \\
\typesetLinecounter{51} & $\Square$ & 2271 & 5 & 31 & Ffaut secondo in terzo ordine & E♯ \typesetKey{F}{3} & {\footnotesize\texttt{D} } &  \\
\typesetLinecounter{52} & $\Square$ & 2368 & 5 & 32 & Ffaut secondo in terzo ordine & E♯ \typesetKey{F}{3} & {\footnotesize\texttt{D} } &  \\
\typesetLinecounter{53} & $\Square$ & 2481 & 5 & 33 & Ffaut secondo in terzo ordine & E♯ \typesetKey{F}{3} & {\footnotesize\texttt{D} } &  \\
\typesetLinecounter{54} & $\Square$ & 2868 & 5 & 36 & Ffaut secondo in terzo ordine & E♯ \typesetKey{F}{3} & {\footnotesize\texttt{D} } &  \\
\typesetLinecounter{55} & $\Square$ & 169 & 5 & 9 & Ffaut secondo [Ffaut secondo in terzo ordine] & E♯ \typesetKey{F}{3} & {\footnotesize\texttt{D} \texttt{sh} } &  \\
\typesetLinecounter{56} & $\Square$ & 1192 & 5 & 20 & Ffaut grave secondo [Ffaut secondo nel terzo ordine] & E♯ \typesetKey{F}{3} & {\footnotesize\texttt{D} \texttt{sh} } &  \\
\typesetLinecounter{57} & $\Square$ & 1193 & 5 & 20 & Ffaut grave secondo [Ffaut secondo nel terzo ordine] & E♯ \typesetKey{F}{3} & {\footnotesize\texttt{D} \texttt{sh} } &  \\
\typesetLinecounter{58} & $\Square$ & 1213 & 5 & 20 & Ffaut gravissimo terzo [Ffaut secondo nel terzo ordine] & E♯ \typesetKey{F}{3} & {\footnotesize\texttt{D} \texttt{sh} } &  \\
\typesetLinecounter{59} & $\Square$ & 1214 & 5 & 20 & medesimo Ffaut terzo grave [Ffaut secondo nel terzo ordine] & E♯ \typesetKey{F}{3} & {\footnotesize\texttt{D} \texttt{sh} } &  \\
\typesetLinecounter{60} & $\Square$ & 1235 & 5 & 20 & Ffaut secondo grave [Ffaut secondo nel terzo ordine] & E♯ \typesetKey{F}{3} & {\footnotesize\texttt{D} \texttt{sh} } &  \\
\typesetLinecounter{61} & $\Square$ & 32 & 5 & 8 & Cfaut terzo in quarto ordine & Ċ \typesetKey{C}{4} & {\footnotesize\texttt{D} } &  \\
\typesetLinecounter{62} & $\Square$ & 42 & 5 & 8 & terzo Cfaut in quarto ordine & Ċ \typesetKey{C}{4} & {\footnotesize\texttt{D} } &  \\
\typesetLinecounter{63} & $\Square$ & 47 & 5 & 8 & terzo Cfaut posto nel quarto ordine & Ċ \typesetKey{C}{4} & {\footnotesize\texttt{D} } &  \\
\typesetLinecounter{64} & $\Square$ & 59 & 5 & 8 & Csolfaut terzo in quarto ordine & Ċ \typesetKey{C}{4} & {\footnotesize\texttt{D} } &  \\
\typesetLinecounter{65} & $\Square$ & 250 & 5 & 10 & Csolfaut terzo in quarto ordine & Ċ \typesetKey{C}{4} & {\footnotesize\texttt{D} } &  \\
\typesetLinecounter{66} & $\Square$ & 326 & 5 & 11 & Cfaut quarto & Ċ \typesetKey{C}{4} & {\footnotesize\texttt{D} } &  \\
\typesetLinecounter{67} & $\Square$ & 1201 & 5 & 20 & Cfaut terzo in quarto ordine & Ċ \typesetKey{C}{4} & {\footnotesize\texttt{D} } &  \\
\typesetLinecounter{68} & $\Square$ & 1441 & 5 & 22 & Csolfaut terzo in quarto ordine & Ċ \typesetKey{C}{4} & {\footnotesize\texttt{D} } &  \\
\typesetLinecounter{69} & $\Square$ & 1728 & 5 & 25 & Csolfaut terzo in quarto ordine & Ċ \typesetKey{C}{4} & {\footnotesize\texttt{D} } &  \\
\typesetLinecounter{70} & $\Square$ & 2444 & 5 & 33 & Cfaut [terzo in quarto ordine] & Ċ \typesetKey{C}{4} & {\footnotesize\texttt{D} } & Ergänzung aus dem Kontext eindeutig ersichtlich. \\
\typesetLinecounter{71} & $\Square$ & 2446 & 5 & 33 & Csolfaut terzo in quarto ordine & Ċ \typesetKey{C}{4} & {\footnotesize\texttt{D} } &  \\
\typesetLinecounter{72} & $\Square$ & 2472 & 5 & 33 & Cfaut terzo in quarto ordine & Ċ \typesetKey{C}{4} & {\footnotesize\texttt{D} } &  \\
\typesetLinecounter{73} & $\Square$ & 2473 & 5 & 33 & esso Cfaut [terzo in quarto ordine] & Ċ \typesetKey{C}{4} & {\footnotesize\texttt{D} } &  \\
\typesetLinecounter{74} & $\Square$ & 2493 & 5 & 33 & Csolfaut terzo in quarto ordine & Ċ \typesetKey{C}{4} & {\footnotesize\texttt{D} } &  \\
\typesetLinecounter{75} & $\Square$ & 2494 & 5 & 33 & Csolfaut terzo in quarto ordine & Ċ \typesetKey{C}{4} & {\footnotesize\texttt{D} } &  \\
\typesetLinecounter{76} & $\Square$ & 2495 & 5 & 33 & Cfaut terzo in quarto ordine & Ċ \typesetKey{C}{4} & {\footnotesize\texttt{D} } &  \\
\typesetLinecounter{77} & $\Square$ & 2753 & 5 & 36 & Cfaut terzo in quarto ordine & Ċ \typesetKey{C}{4} & {\footnotesize\texttt{D} } &  \\
\typesetLinecounter{78} & $\Square$ & 333 & 5 & 11 & Csolfaut terzo [in quarto ordine] & Ċ \typesetKey{C}{4} & {\footnotesize\texttt{D} \texttt{sh} } &  \\
\typesetLinecounter{79} & $\Square$ & 508 & 5 & 12 & Cfaut quarto [Cfaut terzo in quarto ordine] & Ċ \typesetKey{C}{4} & {\footnotesize\texttt{D} \texttt{sh} } &  \\
\typesetLinecounter{80} & $\Square$ & 524 & 5 & 12 & Csolfaut acuto quarto [Csolfaut terzo in quarto ordine] & Ċ \typesetKey{C}{4} & {\footnotesize\texttt{D} \texttt{sh} } &  \\
\typesetLinecounter{81} & $\Square$ & 595 & 5 & 13 & Cfaut quarto [Cfaut terzo in quarto ordine] & Ċ \typesetKey{C}{4} & {\footnotesize\texttt{D} \texttt{sh} } & Verkürzung von Cfaut terzo in quarto ordine. Es könnte auch \emph{Cfaut quarto in sesto ordine} sein, was aber das falsche Intervall ergeben würde: Aʼ-Cʼ ist eine \emph{sesta maggiore}, keine \emph{sesta minore} oder \emph{sesta più di minore}. \\
\typesetLinecounter{82} & $\Square$ & 599 & 5 & 13 & Csolfaut quarto [Csolfaut terzo in quarto ordine] & Ċ \typesetKey{C}{4} & {\footnotesize\texttt{D} \texttt{sh} } &  \\
\typesetLinecounter{83} & $\Square$ & 935 & 5 & 17 & Cfaut quarto [Cfaut terzo nel quarto ordine] & Ċ \typesetKey{C}{4} & {\footnotesize\texttt{D} \texttt{sh} } &  \\
\typesetLinecounter{84} & $\Square$ & 1303 & 5 & 21 & Csolfaut terzo [Csolfaut terzo in quarto ordine] & Ċ \typesetKey{C}{4} & {\footnotesize\texttt{D} \texttt{sh} } &  \\
\typesetLinecounter{85} & $\Square$ & 1342 & 5 & 21 & Csolfaut terzo [Csolfaut terzo in quarto ordine] & Ċ \typesetKey{C}{4} & {\footnotesize\texttt{D} \texttt{sh} } &  \\
\typesetLinecounter{86} & $\Square$ & 1594 & 5 & 24 & Csolfaut quarto [Csolfaut terzo in quarto ordine] & Ċ \typesetKey{C}{4} & {\footnotesize\texttt{D} \texttt{sh} } &  \\
\typesetLinecounter{87} & $\Square$ & 1631 & 5 & 24 & Csolfaut quarto [Csolfaut terzo in quarto ordine] & Ċ \typesetKey{C}{4} & {\footnotesize\texttt{D} \texttt{sh} } &  \\
\typesetLinecounter{88} & $\Square$ & 1693 & 5 & 25 & Csolfaut quarto [Csolfaut terzo in quarto ordine] & Ċ \typesetKey{C}{4} & {\footnotesize\texttt{D} \texttt{sh} } &  \\
\typesetLinecounter{89} & $\Square$ & 2443 & 5 & 33 & Csolfaut quarto [Csolfaut terzo in quarto ordine] & Ċ \typesetKey{C}{4} & {\footnotesize\texttt{D} \texttt{sh} } &  \\
\typesetLinecounter{90} & $\Square$ & 22 & 5 & 8 & Ffaut terzo in quarto ordine & Ḟ \typesetKey{F}{4} & {\footnotesize\texttt{D} } &  \\
\typesetLinecounter{91} & $\Square$ & 72 & 5 & 8 & Ffaut terzo, nel quarto ordine & Ḟ \typesetKey{F}{4} & {\footnotesize\texttt{D} } &  \\
\typesetLinecounter{92} & $\Square$ & 261 & 5 & 10 & Ffaut terzo in quarto ordine & Ḟ \typesetKey{F}{4} & {\footnotesize\texttt{D} } &  \\
\typesetLinecounter{93} & $\Square$ & 614 & 5 & 13 & Ffaut terzo in quarto ordine & Ḟ \typesetKey{F}{4} & {\footnotesize\texttt{D} } &  \\
\typesetLinecounter{94} & $\Square$ & 1801 & 5 & 26 & Ffaut terzo in quarto ordine & Ḟ \typesetKey{F}{4} & {\footnotesize\texttt{D} } &  \\
\typesetLinecounter{95} & $\Square$ & 1812 & 5 & 26 & Ffaut terzo in quarto ordine & Ḟ \typesetKey{F}{4} & {\footnotesize\texttt{D} } &  \\
\typesetLinecounter{96} & $\Square$ & 2008 & 5 & 28 & Ffaut terzo in quarto ordine & Ḟ \typesetKey{F}{4} & {\footnotesize\texttt{D} } &  \\
\typesetLinecounter{97} & $\Square$ & 2195 & 5 & 30 & Ffaut terzo in quarto ordine & Ḟ \typesetKey{F}{4} & {\footnotesize\texttt{D} } &  \\
\typesetLinecounter{98} & $\Square$ & 2743 & 5 & 36 & Ffaut terzo in quarto ordine & Ḟ \typesetKey{F}{4} & {\footnotesize\texttt{D} } &  \\
\typesetLinecounter{99} & $\Square$ & 316 & 5 & 11 & Ffaut quarto [Ffaut terzo in quarto ordine] & Ḟ \typesetKey{F}{4} & {\footnotesize\texttt{D} \texttt{sh} } &  \\
\typesetLinecounter{100} & $\Square$ & 343 & 5 & 11 & Ffaut quarto [Ffaut terzo in quarto ordine] & Ḟ \typesetKey{F}{4} & {\footnotesize\texttt{D} \texttt{sh} } &  \\
\typesetLinecounter{101} & $\Square$ & 498 & 5 & 12 & Ffaut quarto [Ffaut terzo in quarto ordine] & Ḟ \typesetKey{F}{4} & {\footnotesize\texttt{D} \texttt{sh} } &  \\
\typesetLinecounter{102} & $\Square$ & 534 & 5 & 12 & Ffaut quarto [Ffaut terzo in quarto ordine] & Ḟ \typesetKey{F}{4} & {\footnotesize\texttt{D} \texttt{sh} } &  \\
\typesetLinecounter{103} & $\Square$ & 585 & 5 & 13 & Ffaut quarto [Ffaut terzo in quarto ordine] & Ḟ \typesetKey{F}{4} & {\footnotesize\texttt{D} \texttt{sh} } &  \\
\typesetLinecounter{104} & $\Square$ & 2092 & 5 & 29 & Ffaut grave quarto [Ffaut terzo in quarto ordine] & Ḟ \typesetKey{F}{4} & {\footnotesize\texttt{D} \texttt{sh} } &  \\
\typesetLinecounter{105} & $\Square$ & 2099 & 5 & 29 & Ffaut quarto [Ffaut terzo in quarto ordine] & Ḟ \typesetKey{F}{4} & {\footnotesize\texttt{D} \texttt{sh} } &  \\
\typesetLinecounter{106} & $\Square$ & 2180 & 5 & 30 & Ffaut quarto [Ffaut terzo in quarto ordine] & Ḟ \typesetKey{F}{4} & {\footnotesize\texttt{D} \texttt{sh} } &  \\
\typesetLinecounter{107} & $\Square$ & 2458 & 5 & 33 & Ffaut grave quarto [Ffaut terzo in quarto ordine] & Ḟ \typesetKey{F}{4} & {\footnotesize\texttt{D} \texttt{sh} } &  \\
\typesetLinecounter{108} & $\Square$ & 3056 & 5 & 38 & Ffaut quarto [Ffaut terzo in quarto ordine] & Ḟ \typesetKey{F}{4} & {\footnotesize\texttt{D} \texttt{sh} } &  \\

\bottomrule
\end{longtable}
\end{center}
\end{document}
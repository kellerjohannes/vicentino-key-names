%% Auto-generated file: 2024-09-05T21:11:22.000000+02:00
\documentclass[10pt,landscape,DIV=17,a4paper]{scrartcl}
\usepackage[utf8]{inputenc}
\usepackage[T1]{fontenc}
\usepackage[ngerman]{babel}
\usepackage{graphicx}
\usepackage{longtable}
\usepackage{wrapfig}
\usepackage{tipa}
\usepackage{array}
\usepackage{relsize}
\usepackage{amssymb}
\usepackage{mathtools}
\usepackage{wasysym}
\usepackage{booktabs}
\usepackage{soul}
\usepackage{titling}
\usepackage{tikz}
\setcounter{secnumdepth}{0}

\usepackage{newunicodechar}
\newunicodechar{♮}{$\natural$}
\newunicodechar{♭}{$\flat$}
\newunicodechar{♯}{$\sharp$}
\newunicodechar{➚}{{\small$\nearrow$}}
\newunicodechar{➘}{{\small$\searrow$}}
\newunicodechar{Ȧ}{\.A}
\newunicodechar{Ḃ}{\.B}
\newunicodechar{Ċ}{\.C}
\newunicodechar{Ḋ}{\.D}
\newunicodechar{Ė}{\.E}
\newunicodechar{Ḟ}{\.F}
\newunicodechar{Ġ}{\.G}
\newunicodechar{ʼ}{'}

\def\nsharp#1{#1$\sharp$}
\def\nflat#1{#1$\flat$}
\def\nnatural#1{#1$\natural$}
\def\ndot#1{\.{#1}}
\def\nnaturaldot#1{\.{#1}$\natural$}
\def\ncomma#1{\'{#1}}
\def\nnaturalcomma#1{\'{#1}$\natural$}
\def\nflatdot#1{\.{#1}$\flat$}
\def\nsharpdot#1{\.{#1}$\sharp$}


%% This is used for a thighter box around key names
\setlength\fboxsep{1.2pt}

\def\typesetInterval#1#2#3{\small{$\lvert$#1#2#3$\rvert$}}
\def\typesetKey#1#2{\fbox{\footnotesize{\textsc{#1#2}}}}
\def\typesetLinecounter#1{\tiny{\textsc{#1}}}
\def\typesetTag#1{\texttt{#1}}

\renewcommand*{\maketitle}{\noindent%
\parbox{\dimexpr\linewidth-2\fboxsep}{\centering%
\fontsize{20}{24}\selectfont\sffamily\bfseries\thetitle\\[1ex]%
\fontsize{12}{14}\selectfont\centering\today\hspace{1cm}\theauthor}}


\newcolumntype{C}[1]{>{\centering\arraybackslash}p{#1}}

\renewcommand{\arraystretch}{1.3}
\author{Johannes Keller}
\date{\today}

\title{Inventar der \typesetTag{shorthand-Notation}\\\relsize{-3}Berücksichtigt Tastennamen, Intervalle und Noten der Kapitel b5-c8 bis b5-c38}

\begin{document}

\maketitle

\begin{center}

\vspace{3ex}

{\large{Sämtliche Probanden mit dem \typesetTag{:regular-shorthand}-tag.}}

\vspace{2ex}

{\footnotesize{Legende:
\#~Zeilennummerierung,
T~Objekttyp ($\Square$~Taste, $\CIRCLE$~Note, {\tiny$\Square$$\Square$}~Intervall zwischen Tasten, {\tiny$\CIRCLE$$\CIRCLE$}~Intervall zwischen Noten),
I~Objekt-ID,
B~\emph{libro},
C~\emph{capitolo},
»\texttt{D}«~\texttt{:diplomatic},
»\texttt{sh}«~\texttt{:regular-shorthand},
»\texttt{C}«~\texttt{:obvious-correction},
»\texttt{R}«~\texttt{:recommended-correction},
»\texttt{om}«~\texttt{:omitted-text},
»\texttt{extd}«~\texttt{:extended-key},
»\texttt{qs}«~\texttt{:quintenschaukel},
»\texttt{p}«~\texttt{:propinqua},
»\texttt{ip}«~\texttt{:inverse-propinqua},
»\texttt{$\neg$ip}«~\texttt{:avoid-inverse-propinqua},
»\texttt{pp}«~\texttt{:propinquissima},
»\texttt{ipp}«~\texttt{:inverse-propinquissima},
»\texttt{$\neg$ipp}«~\texttt{:avoid-inverse-propinquissima},
»\texttt{ex}«~\texttt{:exotic},
»\texttt{$\neg$ex}«~\texttt{:avoid-exotic},
»\texttt{{\small\fbox{7}}}«~\texttt{:septimal},
Skala der Intervallgrössen: Markierungen für 1:1 81:80, 128:125, 6:5, 5:4, 3:2, 8:5, 5:3 und 2:1.
}}

\begin{longtable}{p{1.5mm}C{1.5mm}p{4.5mm}p{1mm}p{2mm}p{6.5cm}p{15mm}p{1cm}p{11cm}}

\toprule
\# &
\emph{T} &
\emph{I} &
\emph{B} &
\emph{C} &
\emph{Name (normalisierte Orthographie)} &
&
\emph{Tags} &
\emph{Kommentar}\\
\midrule
\endhead


\typesetLinecounter{1} & $\Square$ & 46 & 5 & 8 & Cfaut secondo [Cfaut secondo in terzo ordine] & B♯ \typesetKey{C}{3} & {\footnotesize\texttt{D} \texttt{sh} } & Reguläre Verkürzung von \emph{Cfaut secondo in terzo ordine}. \\
\typesetLinecounter{2} & $\Square$ & 169 & 5 & 9 & Ffaut secondo [Ffaut secondo in terzo ordine] & E♯ \typesetKey{F}{3} & {\footnotesize\texttt{D} \texttt{sh} } &  \\
\typesetLinecounter{3} & $\Square$ & 316 & 5 & 11 & Ffaut quarto [Ffaut terzo in quarto ordine] & Ḟ \typesetKey{F}{4} & {\footnotesize\texttt{D} \texttt{sh} } &  \\
\typesetLinecounter{4} & $\Square$ & 333 & 5 & 11 & Csolfaut terzo [in quarto ordine] & Ċ \typesetKey{C}{4} & {\footnotesize\texttt{D} \texttt{sh} } &  \\
\typesetLinecounter{5} & $\Square$ & 343 & 5 & 11 & Ffaut quarto [Ffaut terzo in quarto ordine] & Ḟ \typesetKey{F}{4} & {\footnotesize\texttt{D} \texttt{sh} } &  \\
\typesetLinecounter{6} & $\Square$ & 498 & 5 & 12 & Ffaut quarto [Ffaut terzo in quarto ordine] & Ḟ \typesetKey{F}{4} & {\footnotesize\texttt{D} \texttt{sh} } &  \\
\typesetLinecounter{7} & $\Square$ & 508 & 5 & 12 & Cfaut quarto [Cfaut terzo in quarto ordine] & Ċ \typesetKey{C}{4} & {\footnotesize\texttt{D} \texttt{sh} } &  \\
\typesetLinecounter{8} & $\Square$ & 524 & 5 & 12 & Csolfaut acuto quarto [Csolfaut terzo in quarto ordine] & Ċ \typesetKey{C}{4} & {\footnotesize\texttt{D} \texttt{sh} } &  \\
\typesetLinecounter{9} & $\Square$ & 534 & 5 & 12 & Ffaut quarto [Ffaut terzo in quarto ordine] & Ḟ \typesetKey{F}{4} & {\footnotesize\texttt{D} \texttt{sh} } &  \\
\typesetLinecounter{10} & $\Square$ & 585 & 5 & 13 & Ffaut quarto [Ffaut terzo in quarto ordine] & Ḟ \typesetKey{F}{4} & {\footnotesize\texttt{D} \texttt{sh} } &  \\
\typesetLinecounter{11} & $\Square$ & 595 & 5 & 13 & Cfaut quarto [Cfaut terzo in quarto ordine] & Ċ \typesetKey{C}{4} & {\footnotesize\texttt{D} \texttt{sh} } & Verkürzung von Cfaut terzo in quarto ordine. Es könnte auch \emph{Cfaut quarto in sesto ordine} sein, was aber das falsche Intervall ergeben würde: Aʼ-Cʼ ist eine \emph{sesta maggiore}, keine \emph{sesta minore} oder \emph{sesta più di minore}. \\
\typesetLinecounter{12} & $\Square$ & 599 & 5 & 13 & Csolfaut quarto [Csolfaut terzo in quarto ordine] & Ċ \typesetKey{C}{4} & {\footnotesize\texttt{D} \texttt{sh} } &  \\
\typesetLinecounter{13} & $\Square$ & 935 & 5 & 17 & Cfaut quarto [Cfaut terzo nel quarto ordine] & Ċ \typesetKey{C}{4} & {\footnotesize\texttt{D} \texttt{sh} } &  \\
\typesetLinecounter{14} & $\Square$ & 1028 & 5 & 18 & Cfaut secondo [Cfaut secondo nel terzo ordine] & B♯ \typesetKey{C}{3} & {\footnotesize\texttt{D} \texttt{sh} } &  \\
\typesetLinecounter{15} & $\Square$ & 1192 & 5 & 20 & Ffaut grave secondo [Ffaut secondo nel terzo ordine] & E♯ \typesetKey{F}{3} & {\footnotesize\texttt{D} \texttt{sh} } &  \\
\typesetLinecounter{16} & $\Square$ & 1193 & 5 & 20 & Ffaut grave secondo [Ffaut secondo nel terzo ordine] & E♯ \typesetKey{F}{3} & {\footnotesize\texttt{D} \texttt{sh} } &  \\
\typesetLinecounter{17} & $\Square$ & 1213 & 5 & 20 & Ffaut gravissimo terzo [Ffaut secondo nel terzo ordine] & E♯ \typesetKey{F}{3} & {\footnotesize\texttt{D} \texttt{sh} } &  \\
\typesetLinecounter{18} & $\Square$ & 1214 & 5 & 20 & medesimo Ffaut terzo grave [Ffaut secondo nel terzo ordine] & E♯ \typesetKey{F}{3} & {\footnotesize\texttt{D} \texttt{sh} } &  \\
\typesetLinecounter{19} & $\Square$ & 1224 & 5 & 20 & Csolfaut secondo [Csolfaut secondo nel terzo ordine] & B♯ \typesetKey{C}{3} & {\footnotesize\texttt{D} \texttt{sh} } &  \\
\typesetLinecounter{20} & $\Square$ & 1235 & 5 & 20 & Ffaut secondo grave [Ffaut secondo nel terzo ordine] & E♯ \typesetKey{F}{3} & {\footnotesize\texttt{D} \texttt{sh} } &  \\
\typesetLinecounter{21} & $\Square$ & 1303 & 5 & 21 & Csolfaut terzo [Csolfaut terzo in quarto ordine] & Ċ \typesetKey{C}{4} & {\footnotesize\texttt{D} \texttt{sh} } &  \\
\typesetLinecounter{22} & $\Square$ & 1307 & 5 & 21 & Csolfaut secondo [Csolfaut secondo in terzo ordine] & B♯ \typesetKey{C}{3} & {\footnotesize\texttt{D} \texttt{sh} } &  \\
\typesetLinecounter{23} & $\Square$ & 1342 & 5 & 21 & Csolfaut terzo [Csolfaut terzo in quarto ordine] & Ċ \typesetKey{C}{4} & {\footnotesize\texttt{D} \texttt{sh} } &  \\
\typesetLinecounter{24} & $\Square$ & 1437 & 5 & 22 & Csolfaut secondo [Csolfaut secondo in terzo ordine] & B♯ \typesetKey{C}{3} & {\footnotesize\texttt{D} \texttt{sh} } &  \\
\typesetLinecounter{25} & $\Square$ & 1594 & 5 & 24 & Csolfaut quarto [Csolfaut terzo in quarto ordine] & Ċ \typesetKey{C}{4} & {\footnotesize\texttt{D} \texttt{sh} } &  \\
\typesetLinecounter{26} & $\Square$ & 1598 & 5 & 24 & Csolfaut sesto [Csolfaut quarto in sesto ordine] & Cʼ \typesetKey{C}{6} & {\footnotesize\texttt{D} \texttt{sh} \texttt{extd} } &  \\
\typesetLinecounter{27} & $\Square$ & 1631 & 5 & 24 & Csolfaut quarto [Csolfaut terzo in quarto ordine] & Ċ \typesetKey{C}{4} & {\footnotesize\texttt{D} \texttt{sh} } &  \\
\typesetLinecounter{28} & $\Square$ & 1635 & 5 & 24 & Csolfaut sesto [Csolfaut quarto in sesto ordine] & Cʼ \typesetKey{C}{6} & {\footnotesize\texttt{D} \texttt{extd} \texttt{sh} } &  \\
\typesetLinecounter{29} & $\Square$ & 1693 & 5 & 25 & Csolfaut quarto [Csolfaut terzo in quarto ordine] & Ċ \typesetKey{C}{4} & {\footnotesize\texttt{D} \texttt{sh} } &  \\
\typesetLinecounter{30} & $\Square$ & 1724 & 5 & 25 & Csolfaut secondo [Csolfaut secondo in terzo ordine] & B♯ \typesetKey{C}{3} & {\footnotesize\texttt{D} \texttt{sh} } &  \\
\typesetLinecounter{31} & $\Square$ & 2092 & 5 & 29 & Ffaut grave quarto [Ffaut terzo in quarto ordine] & Ḟ \typesetKey{F}{4} & {\footnotesize\texttt{D} \texttt{sh} } &  \\
\typesetLinecounter{32} & $\Square$ & 2099 & 5 & 29 & Ffaut quarto [Ffaut terzo in quarto ordine] & Ḟ \typesetKey{F}{4} & {\footnotesize\texttt{D} \texttt{sh} } &  \\
\typesetLinecounter{33} & $\Square$ & 2180 & 5 & 30 & Ffaut quarto [Ffaut terzo in quarto ordine] & Ḟ \typesetKey{F}{4} & {\footnotesize\texttt{D} \texttt{sh} } &  \\
\typesetLinecounter{34} & $\Square$ & 2356 & 5 & 32 & Csolfaut secondo [Csolfaut secondo in terzo ordine] & B♯ \typesetKey{C}{3} & {\footnotesize\texttt{D} \texttt{sh} } &  \\
\typesetLinecounter{35} & $\Square$ & 2378 & 5 & 32 & Cfaut secondo [Csolfaut secondo in terzo ordine] & B♯ \typesetKey{C}{3} & {\footnotesize\texttt{D} \texttt{sh} } &  \\
\typesetLinecounter{36} & $\Square$ & 2379 & 5 & 32 & medesimo Cfaut ascendente [Csolfaut secondo in terzo ordine] & B♯ \typesetKey{C}{3} & {\footnotesize\texttt{D} \texttt{sh} } &  \\
\typesetLinecounter{37} & $\Square$ & 2400 & 5 & 32 & Csolfaut secondo [Csolfaut secondo in terzo ordine] & B♯ \typesetKey{C}{3} & {\footnotesize\texttt{D} \texttt{sh} } &  \\
\typesetLinecounter{38} & $\Square$ & 2401 & 5 & 32 & Cfaut secondo [Csolfaut secondo in terzo ordine] & B♯ \typesetKey{C}{3} & {\footnotesize\texttt{D} \texttt{sh} } &  \\
\typesetLinecounter{39} & $\Square$ & 2443 & 5 & 33 & Csolfaut quarto [Csolfaut terzo in quarto ordine] & Ċ \typesetKey{C}{4} & {\footnotesize\texttt{D} \texttt{sh} } &  \\
\typesetLinecounter{40} & $\Square$ & 2445 & 5 & 33 & Csolfaut secondo acuto [Csolfaut secondo in terzo ordine] & B♯ \typesetKey{C}{3} & {\footnotesize\texttt{D} \texttt{sh} } &  \\
\typesetLinecounter{41} & $\Square$ & 2458 & 5 & 33 & Ffaut grave quarto [Ffaut terzo in quarto ordine] & Ḟ \typesetKey{F}{4} & {\footnotesize\texttt{D} \texttt{sh} } &  \\
\typesetLinecounter{42} & $\Square$ & 3056 & 5 & 38 & Ffaut quarto [Ffaut terzo in quarto ordine] & Ḟ \typesetKey{F}{4} & {\footnotesize\texttt{D} \texttt{sh} } &  \\

\bottomrule
\end{longtable}
\end{center}
\end{document}
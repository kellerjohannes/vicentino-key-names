%% Auto-generated file: 2023-08-08T22:42:56.000000+10:00
\documentclass[10pt,landscape,DIV=17,a4paper]{scrartcl}
\usepackage[utf8]{inputenc}
\usepackage[T1]{fontenc}
\usepackage[ngerman]{babel}
\usepackage{graphicx}
\usepackage{longtable}
\usepackage{wrapfig}
\usepackage{tipa}
\usepackage{array}
\usepackage{relsize}
\usepackage{amssymb}
\usepackage{mathtools}
\usepackage{wasysym}
\usepackage{booktabs}
\usepackage{soul}
\usepackage{titling}
\usepackage{tikz}
\setcounter{secnumdepth}{0}

\usepackage{newunicodechar}
\newunicodechar{♮}{$\natural$}
\newunicodechar{♭}{$\flat$}
\newunicodechar{♯}{$\sharp$}
\newunicodechar{➚}{{\small$\nearrow$}}
\newunicodechar{➘}{{\small$\searrow$}}
\newunicodechar{Ȧ}{\.A}
\newunicodechar{Ḃ}{\.B}
\newunicodechar{Ċ}{\.C}
\newunicodechar{Ḋ}{\.D}
\newunicodechar{Ė}{\.E}
\newunicodechar{Ḟ}{\.F}
\newunicodechar{Ġ}{\.G}
\newunicodechar{ʼ}{'}

\def\nsharp#1{#1$\sharp$}
\def\nflat#1{#1$\flat$}
\def\nnatural#1{#1$\natural$}
\def\ndot#1{\.{#1}}
\def\nnaturaldot#1{\.{#1}$\natural$}
\def\ncomma#1{\'{#1}}
\def\nnaturalcomma#1{\'{#1}$\natural$}
\def\nflatdot#1{\.{#1}$\flat$}
\def\nsharpdot#1{\.{#1}$\sharp$}


%% This is used for a thighter box around key names
\setlength\fboxsep{1.2pt}

\def\typesetInterval#1#2#3{\small{$\lvert$#1#2#3$\rvert$}}
\def\typesetKey#1#2{\fbox{\footnotesize{\textsc{#1#2}}}}
\def\typesetLinecounter#1{\tiny{\textsc{#1}}}
\def\typesetTag#1{\texttt{#1}}

\renewcommand*{\maketitle}{\noindent%
\parbox{\dimexpr\linewidth-2\fboxsep}{\centering%
\fontsize{20}{24}\selectfont\sffamily\bfseries\thetitle\\[1ex]%
\fontsize{12}{14}\selectfont\centering\today\hspace{1cm}\theauthor}}


\newcolumntype{C}[1]{>{\centering\arraybackslash}p{#1}}

\renewcommand{\arraystretch}{1.3}
\author{Johannes Keller}
\date{\today}
\subtitle{Berücksichtigt sämtliche Tastennamen, Intervalle und Noten der Kapitel b5-c8 bis b5-c38.}

\title{Inventar sämtlicher Probanden mit alternativen Lesarten\\\relsize{-3}Berücksichtigt Tastennamen, Intervalle und Noten der Kapitel b5-c8 bis b5-c38}

\begin{document}

\maketitle

\begin{center}

\vspace{3ex}

{\large{Sämtliche Tasten, Intervalle und Noten, die nicht ausschliesslich eine einzige Lesart haben. Dies reflektiert keine Entscheidungen, sondern zeigt alle denkbaren Alternativen.}}

\vspace{2ex}

{\footnotesize{Legende:
\#~Zeilennummerierung,
T~Objekttyp ($\Square$~Taste, $\CIRCLE$~Note, {\tiny$\Square$$\Square$}~Intervall zwischen Tasten, {\tiny$\CIRCLE$$\CIRCLE$}~Intervall zwischen Noten),
I~Objekt-ID,
B~\emph{libro},
C~\emph{capitolo},
»\texttt{D}«~\texttt{:diplomatic},
»\texttt{sh}«~\texttt{:regular-shorthand},
»\texttt{C}«~\texttt{:obvious-correction},
»\texttt{R}«~\texttt{:recommended-correction},
»\texttt{om}«~\texttt{:omitted-text},
»\texttt{extd}«~\texttt{:extended-key},
»\texttt{qs}«~\texttt{:quintenschaukel},
»\texttt{p}«~\texttt{:propinqua},
»\texttt{ip}«~\texttt{:inverse-propinqua},
»\texttt{$\neg$ip}«~\texttt{:avoid-inverse-propinqua},
»\texttt{pp}«~\texttt{:propinquissima},
»\texttt{ipp}«~\texttt{:inverse-propinquissima},
»\texttt{$\neg$ipp}«~\texttt{:avoid-inverse-propinquissima},
»\texttt{ex}«~\texttt{:exotic},
»\texttt{$\neg$ex}«~\texttt{:avoid-exotic},
»\texttt{{\small\fbox{7}}}«~\texttt{:septimal},
Skala der Intervallgrössen: Markierungen für 1:1 81:80, 128:125, 6:5, 5:4, 3:2, 8:5, 5:3 und 2:1.
}}

\begin{longtable}{p{1.5mm}C{1.5mm}p{4.5mm}p{1mm}p{2mm}p{6.5cm}p{15mm}p{1cm}p{11cm}}

\toprule
\# &
\emph{T} &
\emph{I} &
\emph{B} &
\emph{C} &
\emph{Name (normalisierte Orthographie)} &
&
\emph{Tags} &
\emph{Kommentar}\\
\midrule
\endhead


\typesetLinecounter{1} & $\Square$ & 19 & 5 & 8 & Gsolreut terzo & G♭ \typesetKey{G}{3} & {\footnotesize\texttt{D} } &  \\
\typesetLinecounter{2} & $\Square$ & 19 & 5 & 8 & Ffaut terzo in quarto ordine & Ḟ \typesetKey{F}{4} & {\footnotesize\texttt{$\neg$ip} } & Korrektur zur Vermeidung einer inversen \emph{propinqua}. \\
\typesetLinecounter{3} & $\Square$ & 35 & 5 & 8 & Cfaut terzo in quarto ordine & Ċ \typesetKey{C}{4} & {\footnotesize\texttt{D} } & Fragwürdige Passage, dieser Tastenname soll ersatzlos gestrichen werden. \\
\typesetLinecounter{4} & $\Square$ & 35 & 5 & 8 & [gestrichen] & -- \typesetKey{--}{--} & {\footnotesize\texttt{C} } & Original: Ċ. Fragwürdige Passage, dieser Tastenname soll ersatzlos gestrichen werden. Dies ist ein folgenloser Eingriff, weil im Text kein Intervall sich auf diese Taste bezieht. \\
\typesetLinecounter{5} & $\Square$ & 138 & 5 & 9 & -- & -- \typesetKey{--}{--} & {\footnotesize\texttt{D} } & Im Original steht hier kein Tastennamen. \\
\typesetLinecounter{6} & $\Square$ & 138 & 5 & 9 & [Dlasolre secondo] & C♯ \typesetKey{D}{2} & {\footnotesize\texttt{C} \texttt{om} } & Wurde ergänzt, im Original steht kein Tastennamen an dieser Stelle. \\
\typesetLinecounter{7} & $\Square$ & 162 & 5 & 9 & Ffaut secondo in terzo ordine & E♯ \typesetKey{F}{3} & {\footnotesize\texttt{D} } &  \\
\typesetLinecounter{8} & $\Square$ & 162 & 5 & 9 & Elami terzo & D♯ \typesetKey{E}{3} & {\footnotesize\texttt{C} } & Original: »Ffaut secondo in terzo ordine«. Diese Taste soll als Quinte über G♯ funktionieren, muss deshalb \emph{Elami terzo} [D♯] sein. Nebenbemerkung: G♯-E♭ ist die Wolfsquinte,möglicherweise hat das zum Denkfehler von Vicentino gefürt. Siehe auch b5-c7, »la quinta d'Alamire secondo, ascenderà \& ritroverà la sua quinta in Elami terzo acuto« und b5-c23, »\& la sua [Elami terzo acuto] sarà in Gsolreut secondo [recte: Alamire secondo]«. \\
\typesetLinecounter{9} & $\CIRCLE$ & 196 & 5 & 9 & -- & B♭ & {\footnotesize\texttt{D} } &  \\
\typesetLinecounter{10} & $\CIRCLE$ & 196 & 5 & 9 & -- & Ḃ♭ & {\footnotesize\texttt{C} } & Original: B♭. \\
\typesetLinecounter{11} & {\tiny$\Square$$\Square$} & 254 & 5 & 10 & sesta maggiore & \typesetInterval{221}{➚}{255} & {\footnotesize\texttt{D} } &  \\
\typesetLinecounter{12} & {\tiny$\Square$$\Square$} & 254 & 5 & 10 & sesta minore & \typesetInterval{221}{➚}{255} & {\footnotesize\texttt{C} } & Original: »sesta maggiore«. \\
\typesetLinecounter{13} & $\CIRCLE$ & 287 & 5 & 10 & -- & -- & {\footnotesize\texttt{D} } & Fehlt im Notenbeispiel. \\
\typesetLinecounter{14} & $\CIRCLE$ & 287 & 5 & 10 & -- & A♭ & {\footnotesize\texttt{om} } & Fehlt im Original. \\
\typesetLinecounter{15} & {\tiny$\CIRCLE$$\CIRCLE$} & 288 & 5 & 10 & -- & \typesetInterval{268}{➘}{287} & {\footnotesize\texttt{D} } & Fehlt im Notenbeispiel. \\
\typesetLinecounter{16} & {\tiny$\CIRCLE$$\CIRCLE$} & 288 & 5 & 10 & [ottava] & \typesetInterval{268}{➘}{287} & {\footnotesize\texttt{om} } & Fehlt im Original. \\
\typesetLinecounter{17} & $\CIRCLE$ & 307 & 5 & 10 & -- & -- & {\footnotesize\texttt{D} } & Fehlt im Notenbeispiel. \\
\typesetLinecounter{18} & $\CIRCLE$ & 307 & 5 & 10 & -- & A♭ & {\footnotesize\texttt{om} } & Fehlt im Original. \\
\typesetLinecounter{19} & {\tiny$\CIRCLE$$\CIRCLE$} & 308 & 5 & 10 & -- & \typesetInterval{268}{➚}{307} & {\footnotesize\texttt{D} } &  \\
\typesetLinecounter{20} & {\tiny$\CIRCLE$$\CIRCLE$} & 308 & 5 & 10 & [ottava] & \typesetInterval{268}{➚}{307} & {\footnotesize\texttt{R} } & Fehlt im Original. \\
\typesetLinecounter{21} & {\tiny$\Square$$\Square$} & 346 & 5 & 11 & terza maggiore & \typesetInterval{310}{➚}{347} & {\footnotesize\texttt{D} } &  \\
\typesetLinecounter{22} & {\tiny$\Square$$\Square$} & 346 & 5 & 11 & sesta maggiore & \typesetInterval{310}{➚}{347} & {\footnotesize\texttt{C} } & Original: »terza maggiore«. \\
\typesetLinecounter{23} & $\CIRCLE$ & 457 & 5 & 11 & -- & G♯ & {\footnotesize\texttt{D} } &  \\
\typesetLinecounter{24} & $\CIRCLE$ & 457 & 5 & 11 & -- & F♯ & {\footnotesize\texttt{C} } & Original: G♯. \\
\typesetLinecounter{25} & $\Square$ & 526 & 5 & 12 & Cfaut secondo & B♯ \typesetKey{C}{3} & {\footnotesize\texttt{D} } & Problematisch. \\
\typesetLinecounter{26} & $\Square$ & 526 & 5 & 12 & Dlasolre secondo & C♯ \typesetKey{D}{2} & {\footnotesize\texttt{C} } & Original: B♯. Muss korrigiert werden zu Dlasolre secondo [C♯]. Im Notenbeispiel ist die entsprchende Note ebenfalls falsch, als D♭ notiert, sollte aber C♯ sein. \\
\typesetLinecounter{27} & $\Square$ & 536 & 5 & 12 & Ffaut primo & F \typesetKey{F}{1} & {\footnotesize\texttt{D} } & Problematisch. \\
\typesetLinecounter{28} & $\Square$ & 536 & 5 & 12 & Gsolreut secondo & F♯ \typesetKey{G}{2} & {\footnotesize\texttt{C} } & Original: F. Das muss in Gsolreut secondo [F♯] korrigiert werden, was aus dem Kontext eindeutig hervorgeht: Diese Taste soll eine \emph{sesta maggiore propinqua} höher als Ȧ♭ liegen, muss also ein F♯ (Gsolreut secondo). Es könnte ein Druckfehler sein, der beim Setzen entstand, weil eine Zeile höher auch gerade \emph{Ffaut primo} steht. \\
\typesetLinecounter{29} & $\CIRCLE$ & 546 & 5 & 12 & -- & F♯ & {\footnotesize\texttt{D} } &  \\
\typesetLinecounter{30} & $\CIRCLE$ & 546 & 5 & 12 & -- & E♯ & {\footnotesize\texttt{C} } & Original: F♯. \\
\typesetLinecounter{31} & $\CIRCLE$ & 556 & 5 & 12 & -- & Ḃ♮ & {\footnotesize\texttt{D} } &  \\
\typesetLinecounter{32} & $\CIRCLE$ & 556 & 5 & 12 & -- & B♯ & {\footnotesize\texttt{C} } & Original: Ḃ♮. \\
\typesetLinecounter{33} & $\CIRCLE$ & 558 & 5 & 12 & -- & B♮ & {\footnotesize\texttt{D} } &  \\
\typesetLinecounter{34} & $\CIRCLE$ & 558 & 5 & 12 & -- & Ḃ♮ & {\footnotesize\texttt{C} } & Original: B♮. \\
\typesetLinecounter{35} & $\CIRCLE$ & 568 & 5 & 12 & -- & D♭ & {\footnotesize\texttt{D} } &  \\
\typesetLinecounter{36} & $\CIRCLE$ & 568 & 5 & 12 & -- & C♯ & {\footnotesize\texttt{C} } & Original: D♭. \\
\typesetLinecounter{37} & {\tiny$\Square$$\Square$} & 584 & 5 & 13 & terza minore & \typesetInterval{583}{➘}{585} & {\footnotesize\texttt{D} } &  \\
\typesetLinecounter{38} & {\tiny$\Square$$\Square$} & 584 & 5 & 13 & terza più di minore & \typesetInterval{583}{➘}{585} & {\footnotesize\texttt{C} \texttt{p} \texttt{pp} \texttt{ex} } & Original: »terza minore«. Dieses Intervall ist eindeutig grösser als eine \emph{terza minore}, deshalb wurde hier gemäss dem Notenbeispiel der Intervallname angepasst. Streng genommen handelt es sich um eine \emph{terza minore}, die um eine \emph{diesis} und ein \emph{comma} vergrössert ist. \\
\typesetLinecounter{39} & {\tiny$\Square$$\Square$} & 587 & 5 & 13 & terza maggiore & \typesetInterval{588}{➘}{586} & {\footnotesize\texttt{D} \texttt{pp} } &  \\
\typesetLinecounter{40} & {\tiny$\Square$$\Square$} & 587 & 5 & 13 & terza maggiore buonissima & \typesetInterval{588}{➘}{586} & {\footnotesize\texttt{C} \texttt{pp} } & Original: »terza maggiore«. Im Text wird dieser Terz die Eigenschaft »buonissima« zugewiesen. Im Notenbeispiel steht beim entsprechenden Intervall »La terza migliore di maggiore«. \\
\typesetLinecounter{41} & {\tiny$\Square$$\Square$} & 598 & 5 & 13 & terza minore & \typesetInterval{597}{➚}{599} & {\footnotesize\texttt{D} \texttt{p} \texttt{ipp} } &  \\
\typesetLinecounter{42} & {\tiny$\Square$$\Square$} & 598 & 5 & 13 & terza [più di] minore & \typesetInterval{597}{➚}{599} & {\footnotesize\texttt{C} \texttt{p} \texttt{ipp} \texttt{ex} } & Dieses Intervall ist eindeutig grösser als eine \emph{terza minore}, deshalb wurde hier gemäss dem Notenbeispiel der Intervallname angepasst. \\
\typesetLinecounter{43} & {\tiny$\Square$$\Square$} & 602 & 5 & 13 & terza maggiore & \typesetInterval{603}{➚}{604} & {\footnotesize\texttt{D} \texttt{pp} } &  \\
\typesetLinecounter{44} & {\tiny$\Square$$\Square$} & 602 & 5 & 13 & terza più di maggiore & \typesetInterval{603}{➚}{604} & {\footnotesize\texttt{C} \texttt{pp} \texttt{ex} } & Original: »terza maggiore«. Dieses Intervall ist sicher unterschiedlich zu einer \emph{terza maggiore}. Im Originalzusammenhang ist es eine um ein \emph{comma} verkleinerte \emph{terza maggiore} (Aʼ➚C♯). Wird die obere Note zu D♭ korrigiert, handelt es sich um eine \emph{terza maggiore}, die um eine \emph{diesis} vergrössert und um ein \emph{comma} verkleinert ist. Der Intervallname wurde gemäss dem Notenbeispiel übernommen. \\
\typesetLinecounter{45} & $\Square$ & 604 & 5 & 13 & Dsolre secondo acuto & C♯ \typesetKey{D}{2} & {\footnotesize\texttt{D} } &  \\
\typesetLinecounter{46} & $\Square$ & 604 & 5 & 13 & Dsolre terzo & D♭ \typesetKey{D}{3} & {\footnotesize\texttt{R} \texttt{$\neg$ipp} } & Das Intervall \emph{Alamire sesto} zu \emph{Dsolre secondo} ist eine \emph{terza maggiore}, die um ein \emph{comma} verkleinert ist. Da es in diesem Kapitel jedoch um um ein \emph{comma} vergrösserte Intervalle geht, ist es plausibel, an Stelle von \emph{Dsolre secondo} hier \emph{Dsolre terzo} zu idealisieren. Zudem ist die Entsprechung dieses Intervalls im Notenbeispiel mit »terza più die maggiore« bezeichnet. \\
\typesetLinecounter{47} & {\tiny$\Square$$\Square$} & 617 & 5 & 13 & [sesta] più di minore & \typesetInterval{611}{➚}{618} & {\footnotesize\texttt{D} \texttt{p} \texttt{p} \texttt{pp} \texttt{ex} } & Das aufsteigende Intervall von \emph{Alamire sesto} zu \emph{Gsolreut secondo} [A,-F♯] ist eine \emph{terza minore}, die um zwei \emph{diesis} vergrössert und um ein \emph{comma} verkleinert ist. Alternativ kann es auch als eine um ein \emph{comma} verkleinerte \emph{terza maggiore} verstanden werden. \\
\typesetLinecounter{48} & {\tiny$\Square$$\Square$} & 617 & 5 & 13 & sesta per manco di maggiore uno comma & \typesetInterval{611}{➚}{618} & {\footnotesize\texttt{C} \texttt{ipp} \texttt{ex} } & Original: »[sesta] più di minore«. Diese Bezeichnung wird verändert zu \emph{manco di maggiore}, sodass die Abfolge der Statements konsistenter ist. In der Umgebung dieses Intervalls befänden sich dann »sesta per manco di minore uno comma« (Aʼ-F), »sesta più di minore« (Aʼ-Ḟ), »sesta più di minore« [recte: »sesta manco di maggiore]« (Aʼ-F♯) und »sesta più di maggiore« (Aʼ-G♭). \\
\typesetLinecounter{49} & $\CIRCLE$ & 639 & 5 & 13 & -- & Ḋ♭ & {\footnotesize\texttt{D} \texttt{ex} } & Dieses grosse Terz ist um zwei \emph{diesis} vergrössert und um ein \emph{comma} verkleinert. \\
\typesetLinecounter{50} & $\CIRCLE$ & 639 & 5 & 13 & -- & D♭ & {\footnotesize\texttt{R} \texttt{$\neg$ex} } & Original: Ḋ♭. Im Original steht diese Note im Kontext einer als »terza più di maggiore« bezeichneten Terz Aʼ➚Ḋ♭. Diese Terz ist um zwei \emph{diesis} vergrössert und um ein \emph{comma} verkleinert. Diese unwahrscheinliche Konstruktion wird entschärft, indem die obere Note in D♭ korrigiert wird. Die Entsprechung dieses Intervalls im Text lautet Aʼ➚C♯, was einer \emph{terza maggiore} entspricht, die um ein \emph{comma} verkleinert ist. Das steht im Widerspruch mit der Vermittlung von vergrösserten Intervallen, um die es in diesem Kapitel geht. \\
\typesetLinecounter{51} & $\CIRCLE$ & 643 & 5 & 13 & -- & Ė & {\footnotesize\texttt{D} } &  \\
\typesetLinecounter{52} & $\CIRCLE$ & 643 & 5 & 13 & -- & Ḟ & {\footnotesize\texttt{C} } & Original: Ė. \\
\typesetLinecounter{53} & $\Square$ & 654 & 5 & 14 & Elami quarto & Ė \typesetKey{E}{4} & {\footnotesize\texttt{D} } &  \\
\typesetLinecounter{54} & $\Square$ & 654 & 5 & 14 & Elami quinto & Ė♭ \typesetKey{E}{5} & {\footnotesize\texttt{$\neg$ip} } & Original: Ė. Korrigiert zu Ė♭, um die inverse \emph{propinqua} zu vermeiden. \\
\typesetLinecounter{55} & $\Square$ & 658 & 5 & 14 & Elami quinto & Ė♭ \typesetKey{E}{5} & {\footnotesize\texttt{D} } &  \\
\typesetLinecounter{56} & $\Square$ & 658 & 5 & 14 & Elami terzo & D♯ \typesetKey{E}{3} & {\footnotesize\texttt{$\neg$ip} } & Original: Ė♭. Korrektur zu D♯, um eine inverse \emph{propinqua} zu vermeiden. \\
\typesetLinecounter{57} & {\tiny$\Square$$\Square$} & 690 & 5 & 14 & [terza maggiore] propinqua & \typesetInterval{683}{➚}{691} & {\footnotesize\texttt{D} \texttt{pp} } & Dieses Intervall ist eigentlich eine \emph{propinquissima}, wird aber \emph{propinqua} genannt. \\
\typesetLinecounter{58} & {\tiny$\Square$$\Square$} & 690 & 5 & 14 & [terza maggiore] propinquissima & \typesetInterval{683}{➚}{691} & {\footnotesize\texttt{R} \texttt{pp} } & Original: »[terza maggiore] propinqua«. Dieses Intervall ist eindeutig eine \emph{propinquissima}, wird aber \emph{propinqua} genannt. Es handelt um das erste \emph{propinquissima}-Intervall nach dem Kapitel der Einführung des \emph{sesto ordine}. Es liegt nahe, an diesem Punkt der Vermittlung noch keine so strenge Trennung von \emph{propinqua} und \emph{propinquissima} zu erwarten, wie sie in den kommenden Kapiteln zu beobachten ist. \\
\typesetLinecounter{59} & $\Square$ & 691 & 5 & 14 & Bfa♭mi sesto & B♮ʼ \typesetKey{B}{6} & {\footnotesize\texttt{D} \texttt{ex} } &  \\
\typesetLinecounter{60} & $\Square$ & 691 & 5 & 14 & Bfabmi quarto & Ḃ♮ \typesetKey{B}{4} & {\footnotesize\texttt{$\neg$ex} } & G-B, wird im Text als \emph{terza maggiore propinqua} bezeichnet, im Musikbeispiel steht an dieser Stelle jedoch ein G-Ḃ (\emph{Bfa♭mi quarto}). Vor dem Hintergrund des vorhergehenden Kapitels, das die ,-Tasten eingeführt hat ist es plausibel, dass Vicentino nun auch solche komma-modifizierten Terzen in Betracht zieht, allerdings wird es hier inkonsequenterweise als \emph{propinqua} bezeichnet. \\
\typesetLinecounter{61} & $\Square$ & 700 & 5 & 14 & Elami sesto & Eʼ \typesetKey{E}{6} & {\footnotesize\texttt{D} } &  \\
\typesetLinecounter{62} & $\Square$ & 700 & 5 & 14 & Elami quinto & Ė♭ \typesetKey{E}{5} & {\footnotesize\texttt{C} } & Original: Eʼ. Sehr wahrscheinlich ein Fehler, denn \emph{Elami sesto} wird unmittelbar danach als \emph{terza maggiore propinqua} bezeichnet und stimmt dort auch mit dem Notenbeispiel überein. Nur schon aus diesem Grund ist es naheliegend, das \emph{Elami sesto} der vorliegenden Stelle in \emph{Elami quinto} zu korrigieren, denn damit fügt es sich auch in die übliche Sukzession der Intervalle ein. \\
\typesetLinecounter{63} & {\tiny$\Square$$\Square$} & 703 & 5 & 14 & [sesta maggiore] propinqua & \typesetInterval{697}{➚}{704} & {\footnotesize\texttt{D} \texttt{pp} } &  \\
\typesetLinecounter{64} & {\tiny$\Square$$\Square$} & 703 & 5 & 14 & [sesta maggiore] propinquissima & \typesetInterval{697}{➚}{704} & {\footnotesize\texttt{R} \texttt{pp} } & Original: »[sesta maggiore] propinqua«. Analog zu \#690 wird hier argumentiert, dass es sich um frühe \emph{propinquissima}-Intervalle innerhalb der Vermittlung von Vicentino handelt und deshalb die Unterscheidung zwischen \emph{propinqua} und \emph{proopinquissima} noch nicht strikt gehandhabt wird. \\
\typesetLinecounter{65} & $\Square$ & 704 & 5 & 14 & Elami sesto & Eʼ \typesetKey{E}{6} & {\footnotesize\texttt{D} \texttt{ex} } &  \\
\typesetLinecounter{66} & $\Square$ & 704 & 5 & 14 & Elami quarto & Ė \typesetKey{E}{4} & {\footnotesize\texttt{$\neg$ex} } & Original: Eʼ. \\
\typesetLinecounter{67} & $\CIRCLE$ & 717 & 5 & 14 & -- & E♯ & {\footnotesize\texttt{D} } &  \\
\typesetLinecounter{68} & $\CIRCLE$ & 717 & 5 & 14 & -- & D♯ & {\footnotesize\texttt{C} } & Original: E♯. \\
\typesetLinecounter{69} & $\CIRCLE$ & 723 & 5 & 14 & -- & B♭ & {\footnotesize\texttt{D} } &  \\
\typesetLinecounter{70} & $\CIRCLE$ & 723 & 5 & 14 & -- & Ḃ♭ & {\footnotesize\texttt{C} } & Original: B♭. \\
\typesetLinecounter{71} & $\CIRCLE$ & 727 & 5 & 14 & -- & Ȧ & {\footnotesize\texttt{D} } &  \\
\typesetLinecounter{72} & $\CIRCLE$ & 727 & 5 & 14 & -- & A♯ & {\footnotesize\texttt{C} } & Original: Ȧ. \\
\typesetLinecounter{73} & $\CIRCLE$ & 747 & 5 & 14 & -- & Eʼ & {\footnotesize\texttt{D} \texttt{ex} } &  \\
\typesetLinecounter{74} & $\CIRCLE$ & 747 & 5 & 14 & -- & Ė & {\footnotesize\texttt{$\neg$ex} } & Original: Eʼ. \\
\typesetLinecounter{75} & $\Square$ & 756 & 5 & 15 & Elami quinto & Ė♭ \typesetKey{E}{5} & {\footnotesize\texttt{D} } &  \\
\typesetLinecounter{76} & $\Square$ & 756 & 5 & 15 & Dsolre quarto & Ḋ \typesetKey{D}{4} & {\footnotesize\texttt{C} } & Original: Ė♭. \\
\typesetLinecounter{77} & {\tiny$\Square$$\Square$} & 769 & 5 & 15 & [sesta maggiore] propinqua & \typesetInterval{752}{➘}{770} & {\footnotesize\texttt{D} \texttt{ipp} } &  \\
\typesetLinecounter{78} & {\tiny$\Square$$\Square$} & 769 & 5 & 15 & [sesta maggiore] propinquissima & \typesetInterval{752}{➘}{770} & {\footnotesize\texttt{R} \texttt{ipp} } & Original: »[sesta maggiore] propinqua. Dieses Intervall ist eine inverse \emph{propinquissima} und soll deshalb entsprechend benannt werden. Wie auch bei \#690 handelt es sich um einen frühen Fall einer \emph{propinquissima} innerhalb der Chronologie der Vermittlung von Vicentino, weshalb hier die Unterscheidung zwischen \emph{propinqua} und \emph{propinquissima} noch nicht ausgeprägt ist. \\
\typesetLinecounter{79} & $\Square$ & 776 & 5 & 15 & Bfa♭mi terzo & A♯ \typesetKey{B}{3} & {\footnotesize\texttt{D} } &  \\
\typesetLinecounter{80} & $\Square$ & 776 & 5 & 15 & Alamire primo & A \typesetKey{A}{1} & {\footnotesize\texttt{C} } & Original: A♯. Diese und die folgenden Korrekturen sind nötig, weil Vicentino möglicherweise von einer anderen Stammtaste ausging, nämlich \emph{Gsolreut terzo} [G♭] an Stelle von \emph{Gsolreut secondo} [F♯]. Das betrifft auch \#778, \#780 und 782. \\
\typesetLinecounter{81} & $\Square$ & 778 & 5 & 15 & Bfa♭mi secondo & B♭ \typesetKey{B}{2} & {\footnotesize\texttt{D} } &  \\
\typesetLinecounter{82} & $\Square$ & 778 & 5 & 15 & Alamire quarto & Ȧ \typesetKey{A}{4} & {\footnotesize\texttt{C} } & Original: B♭. \\
\typesetLinecounter{83} & $\Square$ & 780 & 5 & 15 & Bfa♭ primo & B♮ \typesetKey{B}{1} & {\footnotesize\texttt{D} } &  \\
\typesetLinecounter{84} & $\Square$ & 780 & 5 & 15 & Bfa♭mi terzo & A♯ \typesetKey{B}{3} & {\footnotesize\texttt{C} } & Original: B♮. \\
\typesetLinecounter{85} & $\Square$ & 784 & 5 & 15 & Csolfaut acuto primo & C \typesetKey{C}{1} & {\footnotesize\texttt{D} } &  \\
\typesetLinecounter{86} & $\Square$ & 784 & 5 & 15 & Dlasolre secondo & C♯ \typesetKey{D}{2} & {\footnotesize\texttt{C} } & Original: C. \\
\typesetLinecounter{87} & $\CIRCLE$ & 797 & 5 & 15 & -- & E♯ & {\footnotesize\texttt{D} } &  \\
\typesetLinecounter{88} & $\CIRCLE$ & 797 & 5 & 15 & -- & D♯ & {\footnotesize\texttt{C} } & Original: E♯. \\
\typesetLinecounter{89} & $\CIRCLE$ & 799 & 5 & 15 & -- & Ė♭ & {\footnotesize\texttt{D} } &  \\
\typesetLinecounter{90} & $\CIRCLE$ & 799 & 5 & 15 & -- & Ḋ & {\footnotesize\texttt{C} } & Original: Ė♭. \\
\typesetLinecounter{91} & $\Square$ & 838 & 5 & 16 & Gsolreut & G \typesetKey{G}{1} & {\footnotesize\texttt{D} } &  \\
\typesetLinecounter{92} & $\Square$ & 838 & 5 & 16 & Gsolreut [terzo] & G♭ \typesetKey{G}{3} & {\footnotesize\texttt{C} } & Original: Gsolreut. Verkürzung von Gsolreut terzo, erschliesst sich eindeutig aus dem Kontext. \\
\typesetLinecounter{93} & $\Square$ & 844 & 5 & 16 & Elami quarto & Ė \typesetKey{E}{4} & {\footnotesize\texttt{D} } &  \\
\typesetLinecounter{94} & $\Square$ & 844 & 5 & 16 & Dlasolre quarto & Ḋ \typesetKey{D}{4} & {\footnotesize\texttt{C} } & Original: »Elami quarto«. \\
\typesetLinecounter{95} & $\Square$ & 846 & 5 & 16 & Elami primo & E \typesetKey{E}{1} & {\footnotesize\texttt{D} } &  \\
\typesetLinecounter{96} & $\Square$ & 846 & 5 & 16 & Dlasolre primo & D \typesetKey{D}{1} & {\footnotesize\texttt{C} } & Original: »Elami primo« \\
\typesetLinecounter{97} & $\Square$ & 879 & 5 & 16 & Gsolreut quarto & Ġ \typesetKey{G}{4} & {\footnotesize\texttt{D} } &  \\
\typesetLinecounter{98} & $\Square$ & 879 & 5 & 16 & Gsolreut terzo & G♭ \typesetKey{G}{3} & {\footnotesize\texttt{C} } & Original: »Gsolreut quarto«. \\
\typesetLinecounter{99} & $\CIRCLE$ & 884 & 5 & 16 & -- & E♯ & {\footnotesize\texttt{D} } &  \\
\typesetLinecounter{100} & $\CIRCLE$ & 884 & 5 & 16 & -- & D♯ & {\footnotesize\texttt{C} } & Original: E♯. \\
\typesetLinecounter{101} & $\Square$ & 938 & 5 & 17 & Bmi & B♮ \typesetKey{B}{1} & {\footnotesize\texttt{D} } &  \\
\typesetLinecounter{102} & $\Square$ & 938 & 5 & 17 & Bmi quarto & Ḃ♮ \typesetKey{B}{4} & {\footnotesize\texttt{C} } & Original: B♮. \\
\typesetLinecounter{103} & $\Square$ & 956 & 5 & 17 & Csolfaut terzo & Ċ \typesetKey{C}{4} & {\footnotesize\texttt{D} } &  \\
\typesetLinecounter{104} & $\Square$ & 956 & 5 & 17 & Csolfaut secondo in terzo ordine & B♯ \typesetKey{C}{3} & {\footnotesize\texttt{C} } & Original: »Csolfaut terzo«. Es handelt sich hier um eine unregelmässige Abkürzung der Bezeichnung von B♯. \\
\typesetLinecounter{105} & $\CIRCLE$ & 1073 & 5 & 18 & -- & Ȧ♭ & {\footnotesize\texttt{D} } &  \\
\typesetLinecounter{106} & $\CIRCLE$ & 1073 & 5 & 18 & -- & Ḃ♭ & {\footnotesize\texttt{C} } & Original: Ȧ♭. \\
\typesetLinecounter{107} & $\CIRCLE$ & 1075 & 5 & 18 & -- & A♭ & {\footnotesize\texttt{D} } &  \\
\typesetLinecounter{108} & $\CIRCLE$ & 1075 & 5 & 18 & -- & B♭ & {\footnotesize\texttt{C} } & Original: Ȧ♭. \\
\typesetLinecounter{109} & $\CIRCLE$ & 1083 & 5 & 18 & -- & B♯ & {\footnotesize\texttt{D} } &  \\
\typesetLinecounter{110} & $\CIRCLE$ & 1083 & 5 & 18 & -- & A♯ & {\footnotesize\texttt{C} } & Original: B♯. \\
\typesetLinecounter{111} & $\CIRCLE$ & 1097 & 5 & 18 & -- & Ė & {\footnotesize\texttt{D} } &  \\
\typesetLinecounter{112} & $\CIRCLE$ & 1097 & 5 & 18 & -- & Ė♭ & {\footnotesize\texttt{C} } & Original: Ė. \\
\typesetLinecounter{113} & $\CIRCLE$ & 1099 & 5 & 18 & -- & E♯ & {\footnotesize\texttt{D} } &  \\
\typesetLinecounter{114} & $\CIRCLE$ & 1099 & 5 & 18 & -- & E & {\footnotesize\texttt{C} } & Original: E♯. \\
\typesetLinecounter{115} & $\CIRCLE$ & 1156 & 5 & 19 & -- & C♭ & {\footnotesize\texttt{D} } &  \\
\typesetLinecounter{116} & $\CIRCLE$ & 1156 & 5 & 19 & -- & D♭ & {\footnotesize\texttt{C} } & Original: C♭. \\
\typesetLinecounter{117} & $\CIRCLE$ & 1168 & 5 & 19 & -- & A♯ & {\footnotesize\texttt{D} } &  \\
\typesetLinecounter{118} & $\CIRCLE$ & 1168 & 5 & 19 & -- & G♯ & {\footnotesize\texttt{C} } & Original: A♯. \\
\typesetLinecounter{119} & $\Square$ & 1230 & 5 & 20 & Dlasolre quarto & Ḋ \typesetKey{D}{4} & {\footnotesize\texttt{D} \texttt{ip} \texttt{ex} } &  \\
\typesetLinecounter{120} & $\Square$ & 1230 & 5 & 20 & Dlasolre quinto & Ḋ♭ \typesetKey{D}{5} & {\footnotesize\texttt{R} \texttt{$\neg$ip} \texttt{$\neg$ex} } & Original: Ḋ. Korrektur zur Vermeidung dieser Unregelmässigen Verwendung einer Naturseptime. \\
\typesetLinecounter{121} & $\Square$ & 1232 & 5 & 20 & Elami terzo & D♯ \typesetKey{E}{3} & {\footnotesize\texttt{D} \texttt{ex} } & Führt zu einer \emph{propinqua} einer Naturseptime. \\
\typesetLinecounter{122} & $\Square$ & 1232 & 5 & 20 & Dlasolre primo & D \typesetKey{D}{1} & {\footnotesize\texttt{R} \texttt{$\neg$ip} \texttt{$\neg$ex} } & Original: »Elami terzo«. Diese Septime (E♯➚D♯) könnte als \emph{propinqua} von der Naturseptime E♯➚Ḋ gelesen werden, was allerdings eine unwahrscheinliche Lesart ist, der im Text unkommentiert bleibt. \\
\typesetLinecounter{123} & $\Square$ & 1234 & 5 & 20 & Ffaut terzo acuto & Ḟ \typesetKey{F}{4} & {\footnotesize\texttt{D} \texttt{ex} } & Hier handelt es sich um eine unregelmässige Verkürzung, denn der Kontext lässt eindeutig auf ein E♯ (»Ffaut secondo in terzo ordine«) schliessen. \\
\typesetLinecounter{124} & $\Square$ & 1234 & 5 & 20 & Ffaut secondo nel terzo ordine & E♯ \typesetKey{F}{3} & {\footnotesize\texttt{C} } & Original: »Ffaut terzo acuto«. \\
\typesetLinecounter{125} & $\CIRCLE$ & 1256 & 5 & 20 & -- & Ġ♭ & {\footnotesize\texttt{D} } &  \\
\typesetLinecounter{126} & $\CIRCLE$ & 1256 & 5 & 20 & -- & Ġ & {\footnotesize\texttt{C} } & Original: Ġ♭. \\
\typesetLinecounter{127} & $\CIRCLE$ & 1274 & 5 & 20 & -- & Ḋ & {\footnotesize\texttt{D} } &  \\
\typesetLinecounter{128} & $\CIRCLE$ & 1274 & 5 & 20 & -- & Ḋ♭ & {\footnotesize\texttt{C} } & Original: Original: Ḋ. \\
\typesetLinecounter{129} & $\CIRCLE$ & 1276 & 5 & 20 & -- & D♯ & {\footnotesize\texttt{D} } &  \\
\typesetLinecounter{130} & $\CIRCLE$ & 1276 & 5 & 20 & -- & D & {\footnotesize\texttt{C} } & Original: D♯. \\
\typesetLinecounter{131} & $\Square$ & 1419 & 5 & 22 & Elami grave terzo & D♯ \typesetKey{E}{3} & {\footnotesize\texttt{D} } &  \\
\typesetLinecounter{132} & $\Square$ & 1419 & 5 & 22 & Elami grave secondo & E♭ \typesetKey{E}{2} & {\footnotesize\texttt{C} } & Original: D♯. \\
\typesetLinecounter{133} & $\Square$ & 1423 & 5 & 22 & Gsolreut secondo & F♯ \typesetKey{G}{2} & {\footnotesize\texttt{D} } &  \\
\typesetLinecounter{134} & $\Square$ & 1423 & 5 & 22 & Gsolreut quinto & Ġ♭ \typesetKey{G}{5} & {\footnotesize\texttt{$\neg$ip} } &  \\
\typesetLinecounter{135} & $\Square$ & 1511 & 5 & 23 & Gsolreut secondo & F♯ \typesetKey{G}{2} & {\footnotesize\texttt{D} } &  \\
\typesetLinecounter{136} & $\Square$ & 1511 & 5 & 23 & Alamire secondo & G♯ \typesetKey{A}{2} & {\footnotesize\texttt{C} } & Original: F♯. \\
\typesetLinecounter{137} & $\Square$ & 1604 & 5 & 24 & Alamire primo & A \typesetKey{A}{1} & {\footnotesize\texttt{D} } &  \\
\typesetLinecounter{138} & $\Square$ & 1604 & 5 & 24 & Alamire terzo & A♭ \typesetKey{A}{3} & {\footnotesize\texttt{C} } & Original: »Alamire primo«. \\
\typesetLinecounter{139} & $\CIRCLE$ & 1657 & 5 & 24 & -- & G♭ & {\footnotesize\texttt{D} } &  \\
\typesetLinecounter{140} & $\CIRCLE$ & 1657 & 5 & 24 & -- & A♭ & {\footnotesize\texttt{C} } & Original: G♭. \\
\typesetLinecounter{141} & $\Square$ & 1716 & 5 & 25 & Gsolreut terzo & G♭ \typesetKey{G}{3} & {\footnotesize\texttt{D} } &  \\
\typesetLinecounter{142} & $\Square$ & 1716 & 5 & 25 & Gsolreut primo & G \typesetKey{G}{1} & {\footnotesize\texttt{$\neg$ip} } & Original: »Gsolreut terzo«. Korrektur um eine inverse \emph{propinqua} zu vermeiden. \\
\typesetLinecounter{143} & $\Square$ & 1720 & 5 & 25 & Alamire terzo & A♭ \typesetKey{A}{3} & {\footnotesize\texttt{D} } &  \\
\typesetLinecounter{144} & $\Square$ & 1720 & 5 & 25 & Alamire secondo & G♯ \typesetKey{A}{2} & {\footnotesize\texttt{C} } & Original: »Alamire terzo«. \\
\typesetLinecounter{145} & {\tiny$\Square$$\Square$} & 1821 & 5 & 26 & sesta maggiore & \typesetInterval{1808}{➚}{1822} & {\footnotesize\texttt{D} } &  \\
\typesetLinecounter{146} & {\tiny$\Square$$\Square$} & 1821 & 5 & 26 & sesta minore & \typesetInterval{1808}{➚}{1822} & {\footnotesize\texttt{C} } & Original: »sesta maggiore«. \\
\typesetLinecounter{147} & $\Square$ & 1898 & 5 & 27 & Ffaut grave primo & F \typesetKey{F}{1} & {\footnotesize\texttt{D} } &  \\
\typesetLinecounter{148} & $\Square$ & 1898 & 5 & 27 & Elami primo & E \typesetKey{E}{1} & {\footnotesize\texttt{C} } & Original: »Ffaut grave primo«. Offensichtliche Korrektur zu \emph{Elami primo} [E] nötig, denn die \emph{sesta maggiore} unter \emph{Dlasolre secondo} [C♯] liegt in [E]. \emph{Ffaut grave primo} [F] würde das Intervall der übermässigen Quinte ergeben, was sinngemäss bei Vicentino \emph{sesta manca di minore} heissen könnte. \\
\typesetLinecounter{149} & $\Square$ & 1900 & 5 & 27 & Ffaut secondo in terzo ordine & E♯ \typesetKey{F}{3} & {\footnotesize\texttt{D} } &  \\
\typesetLinecounter{150} & $\Square$ & 1900 & 5 & 27 & Elami quinto & Ė♭ \typesetKey{E}{5} & {\footnotesize\texttt{C} } & Original: »Ffaut secondo in terzo ordine«. Offensichtlicher Fehler, denn \emph{Ffaut secondo in terzo ordine} [E♯] ist bereits die \emph{sesta minore} zu [C♯]. Deshalb soll hier zu \emph{Elami quinto} [Ė♭] korrigiert werden, was auch dem Notenbeispiel entspricht und eine \emph{sesta maggiore propinqua} herstellt. \\
\typesetLinecounter{151} & {\tiny$\Square$$\Square$} & 2011 & 5 & 28 & sesta maggiore & \typesetInterval{2000}{➚}{2012} & {\footnotesize\texttt{D} } &  \\
\typesetLinecounter{152} & {\tiny$\Square$$\Square$} & 2011 & 5 & 28 & sesta minore & \typesetInterval{2000}{➚}{2012} & {\footnotesize\texttt{C} } & Original: »sesta maggiore«. \\
\typesetLinecounter{153} & $\CIRCLE$ & 2028 & 5 & 28 & -- & B♯ & {\footnotesize\texttt{D} } &  \\
\typesetLinecounter{154} & $\CIRCLE$ & 2028 & 5 & 28 & -- & A♯ & {\footnotesize\texttt{C} } & Original: B♯. \\
\typesetLinecounter{155} & $\Square$ & 2096 & 5 & 29 & Dsolre primo & D \typesetKey{D}{1} & {\footnotesize\texttt{D} } &  \\
\typesetLinecounter{156} & $\Square$ & 2096 & 5 & 29 & Dsolre quarto & Ḋ \typesetKey{D}{4} & {\footnotesize\texttt{C} } & Original: »Dsolre primo«. \\
\typesetLinecounter{157} & $\CIRCLE$ & 2159 & 5 & 29 & -- & Ḋ & {\footnotesize\texttt{D} } &  \\
\typesetLinecounter{158} & $\CIRCLE$ & 2159 & 5 & 29 & -- & Ḃ♮ & {\footnotesize\texttt{C} } & Original: Ḋ. \\
\typesetLinecounter{159} & $\CIRCLE$ & 2161 & 5 & 29 & -- & D♯ & {\footnotesize\texttt{D} } &  \\
\typesetLinecounter{160} & $\CIRCLE$ & 2161 & 5 & 29 & -- & B♯ & {\footnotesize\texttt{C} } & Original: D♯. \\
\typesetLinecounter{161} & $\CIRCLE$ & 2163 & 5 & 29 & -- & Ḟ & {\footnotesize\texttt{D} } &  \\
\typesetLinecounter{162} & $\CIRCLE$ & 2163 & 5 & 29 & -- & Ḋ & {\footnotesize\texttt{C} } & Original: Ḟ. \\
\typesetLinecounter{163} & $\Square$ & 2178 & 5 & 30 & Gsolreut & G \typesetKey{G}{1} & {\footnotesize\texttt{D} } &  \\
\typesetLinecounter{164} & $\Square$ & 2178 & 5 & 30 & Gsolreut quinto & Ġ♭ \typesetKey{G}{5} & {\footnotesize\texttt{C} } & Original: »Gsolreut«. [quinto] erschliesst sich eindeutig aus dem Kontext. \\
\typesetLinecounter{165} & $\Square$ & 2265 & 5 & 31 & Alamire secondo & G♯ \typesetKey{A}{2} & {\footnotesize\texttt{D} } &  \\
\typesetLinecounter{166} & $\Square$ & 2265 & 5 & 31 & Alamire terzo & A♭ \typesetKey{A}{3} & {\footnotesize\texttt{C} } & Original: »Alamire secondo«. \\
\typesetLinecounter{167} & $\Square$ & 2267 & 5 & 31 & Gsolreut quarto & Ġ \typesetKey{G}{4} & {\footnotesize\texttt{D} } &  \\
\typesetLinecounter{168} & $\Square$ & 2267 & 5 & 31 & Alamire secondo & G♯ \typesetKey{A}{2} & {\footnotesize\texttt{C} } & Original: »Gsolreut quarto«. \\
\typesetLinecounter{169} & $\CIRCLE$ & 2316 & 5 & 31 & -- & G♭ & {\footnotesize\texttt{D} } &  \\
\typesetLinecounter{170} & $\CIRCLE$ & 2316 & 5 & 31 & -- & A♭ & {\footnotesize\texttt{C} } & Original: G♭. \\
\typesetLinecounter{171} & $\Square$ & 2381 & 5 & 32 & Elami secondo & E♭ \typesetKey{E}{2} & {\footnotesize\texttt{D} } &  \\
\typesetLinecounter{172} & $\Square$ & 2381 & 5 & 32 & Elami terzo & D♯ \typesetKey{E}{3} & {\footnotesize\texttt{C} } & Original: E♭. \\
\typesetLinecounter{173} & {\tiny$\Square$$\Square$} & 2396 & 5 & 32 & quinta propinqua & \typesetInterval{2379}{➚}{2397} & {\footnotesize\texttt{D} } &  \\
\typesetLinecounter{174} & {\tiny$\Square$$\Square$} & 2396 & 5 & 32 & sesta maggiore propinqua & \typesetInterval{2379}{➚}{2397} & {\footnotesize\texttt{C} \texttt{p} } & Original: »quinta propinqua«. \\
\typesetLinecounter{175} & $\Square$ & 2399 & 5 & 32 & Csolfaut acuto & C \typesetKey{C}{1} & {\footnotesize\texttt{D} } &  \\
\typesetLinecounter{176} & $\Square$ & 2399 & 5 & 32 & Csolfaut [secondo in terzo ordine] acuto & B♯ \typesetKey{C}{3} & {\footnotesize\texttt{C} } & Original: »Csolfaut acuto«. Tastenbezeichnung ergibt sich eindeutig aus dem Kontext. \\
\typesetLinecounter{177} & $\CIRCLE$ & 2403 & 5 & 32 & -- & Ġ♭ & {\footnotesize\texttt{D} } &  \\
\typesetLinecounter{178} & $\CIRCLE$ & 2403 & 5 & 32 & -- & Ȧ♭ & {\footnotesize\texttt{C} } & Original: Ġ♭. \\
\typesetLinecounter{179} & $\CIRCLE$ & 2405 & 5 & 32 & -- & G♭ & {\footnotesize\texttt{D} } &  \\
\typesetLinecounter{180} & $\CIRCLE$ & 2405 & 5 & 32 & -- & A♭ & {\footnotesize\texttt{C} } & Original: G♭. \\
\typesetLinecounter{181} & $\CIRCLE$ & 2435 & 5 & 32 & -- & G♭ & {\footnotesize\texttt{D} } &  \\
\typesetLinecounter{182} & $\CIRCLE$ & 2435 & 5 & 32 & -- & A♭ & {\footnotesize\texttt{C} } & Original: G♭. \\
\typesetLinecounter{183} & $\CIRCLE$ & 2437 & 5 & 32 & -- & Ġ♭ & {\footnotesize\texttt{D} } &  \\
\typesetLinecounter{184} & $\CIRCLE$ & 2437 & 5 & 32 & -- & Ȧ♭ & {\footnotesize\texttt{C} } & Original: Ġ♭ \\
\typesetLinecounter{185} & $\CIRCLE$ & 2439 & 5 & 32 & -- & G & {\footnotesize\texttt{D} } &  \\
\typesetLinecounter{186} & $\CIRCLE$ & 2439 & 5 & 32 & -- & A & {\footnotesize\texttt{C} } & Original: G. \\
\typesetLinecounter{187} & $\Square$ & 2454 & 5 & 33 & Bfabmi quinto & Ḃ♭ \typesetKey{B}{5} & {\footnotesize\texttt{D} } &  \\
\typesetLinecounter{188} & $\Square$ & 2454 & 5 & 33 & Alamire quinto & Ȧ♭ \typesetKey{A}{5} & {\footnotesize\texttt{C} } & Original: Ḃ♭. \\
\typesetLinecounter{189} & $\Square$ & 2456 & 5 & 33 & Bfabmi secondo & B♭ \typesetKey{B}{2} & {\footnotesize\texttt{D} } &  \\
\typesetLinecounter{190} & $\Square$ & 2456 & 5 & 33 & Alamire terzo & A♭ \typesetKey{A}{3} & {\footnotesize\texttt{C} } & Original: B♭. \\
\typesetLinecounter{191} & $\Square$ & 2487 & 5 & 33 & Alamire terzo & A♭ \typesetKey{A}{3} & {\footnotesize\texttt{D} \texttt{ip} } &  \\
\typesetLinecounter{192} & $\Square$ & 2487 & 5 & 33 & Csolfaut terzo in quarto ordine & A♭ \typesetKey{A}{3} & {\footnotesize\texttt{$\neg$ip} } & Das Interval \emph{Csolfaut terzo in quarto ordine} [Ċ] -- \emph{Alamire terzo} [A♭] ist eine Sexte, die um eine \emph{diesis} verkleinert ist, also \emph{sesta manca di minore} genannt werden könnte. Die Bezeichnung \emph{propinqua} wäre in diesem Sinn unüblich. Da dies ein nur selten erwähntes Intervall ist liegt eine Korrektur zu \emph{Alamire primo} [A] nahe, was einer \emph{sesta minore propinqua} entspricht und mit dem Notenbeispiel übereinstimmt. \\
\typesetLinecounter{193} & $\CIRCLE$ & 2501 & 5 & 33 & -- & Ȧ & {\footnotesize\texttt{D} } &  \\
\typesetLinecounter{194} & $\CIRCLE$ & 2501 & 5 & 33 & -- & A & {\footnotesize\texttt{C} } & Original: Ȧ. \\
\typesetLinecounter{195} & $\CIRCLE$ & 2503 & 5 & 33 & -- & Ġ♭ & {\footnotesize\texttt{D} } &  \\
\typesetLinecounter{196} & $\CIRCLE$ & 2503 & 5 & 33 & -- & Ȧ♭ & {\footnotesize\texttt{C} } & Original: Ġ♭. \\
\typesetLinecounter{197} & $\CIRCLE$ & 2505 & 5 & 33 & -- & G♭ & {\footnotesize\texttt{D} } &  \\
\typesetLinecounter{198} & $\CIRCLE$ & 2505 & 5 & 33 & -- & A♭ & {\footnotesize\texttt{C} } & Original: G♭. \\
\typesetLinecounter{199} & $\CIRCLE$ & 2695 & 5 & 35 & -- & G♯ & {\footnotesize\texttt{D} } &  \\
\typesetLinecounter{200} & $\CIRCLE$ & 2695 & 5 & 35 & -- & F♯ & {\footnotesize\texttt{C} } & Original: G♯. \\
\typesetLinecounter{201} & $\Square$ & 2758 & 5 & 36 & Dsolre terzo & D♭ \typesetKey{D}{3} & {\footnotesize\texttt{D} } &  \\
\typesetLinecounter{202} & $\Square$ & 2758 & 5 & 36 & Dsolre secondo & C♯ \typesetKey{D}{2} & {\footnotesize\texttt{C} } & Original: »Dsolre terzo«. \\
\typesetLinecounter{203} & $\Square$ & 2860 & 5 & 36 & Dsolre secondo & C♯ \typesetKey{D}{2} & {\footnotesize\texttt{D} } &  \\
\typesetLinecounter{204} & $\Square$ & 2860 & 5 & 36 & Dsolre terzo & D♭ \typesetKey{D}{3} & {\footnotesize\texttt{C} } & Original: »Dsolre secondo«. \\
\typesetLinecounter{205} & $\Square$ & 2874 & 5 & 36 & Alamire secondo & G♯ \typesetKey{A}{2} & {\footnotesize\texttt{D} } &  \\
\typesetLinecounter{206} & $\Square$ & 2874 & 5 & 36 & Gsolreut quinto & Ġ♭ \typesetKey{G}{5} & {\footnotesize\texttt{C} } & Original: »Alamire secondo«. \\
\typesetLinecounter{207} & $\Square$ & 2876 & 5 & 36 & Alamire terzo & A♭ \typesetKey{A}{3} & {\footnotesize\texttt{D} } &  \\
\typesetLinecounter{208} & $\Square$ & 2876 & 5 & 36 & Gsolreut primo & G \typesetKey{G}{1} & {\footnotesize\texttt{C} } & Original: »Alamire terzo«. \\
\typesetLinecounter{209} & $\CIRCLE$ & 2892 & 5 & 36 & -- & Ċ♭ & {\footnotesize\texttt{D} } &  \\
\typesetLinecounter{210} & $\CIRCLE$ & 2892 & 5 & 36 & -- & Ḋ♭ & {\footnotesize\texttt{C} } & Original: Ċ♭. \\
\typesetLinecounter{211} & $\CIRCLE$ & 2894 & 5 & 36 & -- & C♭ & {\footnotesize\texttt{D} } &  \\
\typesetLinecounter{212} & $\CIRCLE$ & 2894 & 5 & 36 & -- & D♭ & {\footnotesize\texttt{C} } & Original: C♭. \\
\typesetLinecounter{213} & $\CIRCLE$ & 2912 & 5 & 36 & -- & G♯ & {\footnotesize\texttt{D} } &  \\
\typesetLinecounter{214} & $\CIRCLE$ & 2912 & 5 & 36 & -- & F♯ & {\footnotesize\texttt{C} } & Original: G♯. \\
\typesetLinecounter{215} & $\CIRCLE$ & 2914 & 5 & 36 & -- & A♭ & {\footnotesize\texttt{D} } &  \\
\typesetLinecounter{216} & $\CIRCLE$ & 2914 & 5 & 36 & -- & G♭ & {\footnotesize\texttt{C} } & Original: A♭. \\
\typesetLinecounter{217} & $\CIRCLE$ & 2916 & 5 & 36 & -- & Ȧ♭ & {\footnotesize\texttt{D} } &  \\
\typesetLinecounter{218} & $\CIRCLE$ & 2916 & 5 & 36 & -- & Ġ♭ & {\footnotesize\texttt{C} } & Original: Ȧ♭. \\
\typesetLinecounter{219} & $\CIRCLE$ & 2918 & 5 & 36 & -- & Aʼ & {\footnotesize\texttt{D} } &  \\
\typesetLinecounter{220} & $\CIRCLE$ & 2918 & 5 & 36 & -- & Gʼ & {\footnotesize\texttt{C} } & Original: Aʼ. \\
\typesetLinecounter{221} & $\CIRCLE$ & 2920 & 5 & 36 & -- & Ȧ & {\footnotesize\texttt{D} } &  \\
\typesetLinecounter{222} & $\CIRCLE$ & 2920 & 5 & 36 & -- & G & {\footnotesize\texttt{C} } & Original: Ȧ. \\
\typesetLinecounter{223} & $\Square$ & 2973 & 5 & 37 & Bfabmi acuto & B♮ \typesetKey{B}{1} & {\footnotesize\texttt{D} } &  \\
\typesetLinecounter{224} & $\Square$ & 2973 & 5 & 37 & Bfabmi quarto & Ḃ♮ \typesetKey{B}{4} & {\footnotesize\texttt{C} } & Original: B♮. \\
\typesetLinecounter{225} & $\CIRCLE$ & 3017 & 5 & 37 & -- & B♮ & {\footnotesize\texttt{D} } &  \\
\typesetLinecounter{226} & $\CIRCLE$ & 3017 & 5 & 37 & -- & Ḃ♮ & {\footnotesize\texttt{C} } & Original: B♮. \\
\typesetLinecounter{227} & {\tiny$\Square$$\Square$} & 3029 & 5 & 38 & [terza maggiore] propinquissima & \typesetInterval{3022}{➘}{3030} & {\footnotesize\texttt{D} \texttt{pp} } &  \\
\typesetLinecounter{228} & {\tiny$\Square$$\Square$} & 3029 & 5 & 38 & [terza maggiore] propinqua & \typesetInterval{3022}{➘}{3030} & {\footnotesize\texttt{C} \texttt{p} } & Original: »[terza maggiore] propinquissima«. Eine \emph{terza maggiore propiquissima} existiert an dieser Stelle nicht in der Klaviatur, deshalb soll dieser Intervallname auf jeden Fall als \emph{propinqua} verstanden werden. \\

\bottomrule
\end{longtable}
\end{center}
\end{document}